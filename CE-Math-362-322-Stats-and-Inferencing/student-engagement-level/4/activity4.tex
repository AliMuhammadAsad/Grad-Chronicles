 Define Article %%%%%
\documentclass{article}
%%%%%%%%%%%%%%%%%%%%%

 Using Packages %%%%%
\usepackage{geometry}
\usepackage{graphicx}
\usepackage{amssymb}
\usepackage{amsmath}
\usepackage{amsthm}
\usepackage{empheq}
\usepackage{mdframed}
\usepackage{booktabs}
\usepackage{lipsum}
\usepackage{graphicx}
\usepackage{color}
\usepackage{psfrag}
\usepackage{pgfplots}
\usepackage{bm}
%%%%%%%%%%%%%%%%%%%%%

% Other Settings

%%%%%%%%%%%%%%%%%%%%%%%%%% Page Setting %%%%%%%%%%
\geometry{a4paper}

%%%%%%%%%%%%%%%%%%%%%%%%%% Define some useful colors %%%%%%%%%%%%%%%%%%%%%%%%%%
\definecolor{ocre}{RGB}{243,102,25}
\definecolor{mygray}{RGB}{243,243,244}
\definecolor{deepGreen}{RGB}{26,111,0}
\definecolor{shallowGreen}{RGB}{235,255,255}
\definecolor{deepBlue}{RGB}{61,124,222}
\definecolor{shallowBlue}{RGB}{235,249,255}
%%%%%%%%%%%%%%%%%%%%%

%%%%%%%%%%%%%%%%%%%%%%%%%% Define an orangebox command %%%%%%%%%%%%%%%%%%%%%%%%
\newcommand\orangebox[1]{\fcolorbox{ocre}{mygray}{\hspace{1em}#1\hspace{1em}}}
%%%%%%%%%%%%%%%%%%%%%

%%%%%%%%%%%%%%%%%%%%%%%%%%%% English Environments 
\newtheoremstyle{mytheoremstyle}{3pt}{3pt}{\normalfont}{0cm}{\rmfamily\bfseries}{}{1em}{{\color{black}\thmname{#1}~\thmnumber{#2}}\thmnote{\,--\,#3}}
\newtheoremstyle{myproblemstyle}{3pt}{3pt}{\normalfont}{0cm}{\rmfamily\bfseries}{}{1em}{{\color{black}\thmname{#1}~\thmnumber{#2}}\thmnote{\,--\,#3}}
\theoremstyle{mytheoremstyle}
\newmdtheoremenv[linewidth=1pt,backgroundcolor=shallowGreen,linecolor=deepGreen,leftmargin=0pt,innerleftmargin=20pt,innerrightmargin=20pt,]{theorem}{Theorem}[section]
\theoremstyle{mytheoremstyle}
\newmdtheoremenv[linewidth=1pt,backgroundcolor=shallowBlue,linecolor=deepBlue,leftmargin=0pt,innerleftmargin=20pt,innerrightmargin=20pt,]{definition}{Definition}[section]
\theoremstyle{myproblemstyle}
\newmdtheoremenv[linecolor=black,leftmargin=0pt,innerleftmargin=10pt,innerrightmargin=10pt,]{problem}{Problem}[section]
%%%%%%%%%%%%%%%%%%%%%

%% Plotting Settings 
\usepgfplotslibrary{colorbrewer}
\pgfplotsset{width=8cm,compat=1.9}
%%%%%%%%%%%%%%%%%%%%%

%% Title & Author %%%
\title{Week 3 SEL Activity 4}
\author{Muhammad Meesum Ali Qazalbash}
%%%%%%%%%%%%%%%%%%%%%

\begin{document}
\maketitle

\section{MGF of Bernoulli RV}

MGF of Discrete RVs is given as,
\[M_{X}(t)=\sum_{x\in X}e^{tx}p_{X}(x)\]
For \(X\sim \operatorname{Ber}(;p)\),
\begin{equation}
	\begin{split}
		M_{X}(t)&=e^{t(0)}(1-p)+e^{t(1)}p\\
		M_{X}(t)&=1-p+pe^{t}\\
	\end{split}
\end{equation}

\section{Mean and Variance of Bernoulli RV}

First and second moment of \(X\) would be,
\[\mu_1=\frac{\partial M_{X}(t)}{\partial t}\bigg|_{t=0}=pe^{t}\bigg|_{t=0}=p\]
\[\mu_2=\frac{\partial^2 M_{X}(t)}{\partial t^2}\bigg|_{t=0}=\frac{\partial}{\partial t}\left(pe^{t}\right)\bigg|_{t=0}=pe^{t}\bigg|_{t=0}=p\]
The mean and variance is,
\[\mu_{X}=\mu_1=p\]
\[\sigma_{X}^{2}=\mu_2-\mu_1^2=p-p^2=p(1-p)\]

\section{Mean and Variance of Binomial RV}

Let \(Y\sim \operatorname{B}(;p,n)\). \(Y\) can be written as the sum of multiple Bernoulli RV.
\[Y=\sum_{i=1}^{n}X_i\]
where \(X_i\sim \operatorname{Ber}(;p)\) for \(1\le i\le n\). According to the theorem mentioned in the question.
\[M_{Y}(t)=\prod_{i=1}^{n}M_{X_{i}}(t)\]
We have calculated the MGF of Bernoulli RV.
\[M_{Y}(t)=\prod_{i=1}^{n}\left(1-p+pe^{t}\right)=\left(1-p+pe^{t}\right)^n\]
First and second partial differentail would be,
\[\partial_{t}M_{Y}(t)=npe^{t}\left(1-p+pe^{t}\right)^{n-1}\]
\[\partial_{t}^{2}M_{Y}(t)=npe^{t}\left(1-p+pe^{t}\right)^{n-1}+n(n-1)p^2e^{2t}\left(1-p+pe^{t}\right)^{n-2}\]
\[\partial_{t}^{2}M_{Y}(t)=npe^{t}\left(1-p+pe^{t}\right)^{n-2}\left(1-p+npe^{t}\right)\]
Moments will be,
\[\mu_1=\partial_{t}M_{Y}(t)\big|_{t=0}=np\]
\[\mu_2=\partial_{t}^{2}M_{Y}(t)\big|_{t=0}=np(1-p+np)\]
Mean and Varaince would be,
\[\mu_{Y}=\mu_{1}=np\]
\[\sigma_{Y}^2=\mu_2-\mu_1^2=np(1-p+np)-n^2p^2=np(1-p)\]
\end{document}