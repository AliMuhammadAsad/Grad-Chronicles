 Define Article %%%%%
\documentclass{article}
%%%%%%%%%%%%%%%%%%%%%

 Using Packages %%%%%
\usepackage{geometry}
\usepackage{graphicx}
\usepackage{amssymb}
\usepackage{amsmath}
\usepackage{amsthm}
\usepackage{empheq}
\usepackage{mdframed}
\usepackage{booktabs}
\usepackage{lipsum}
\usepackage{graphicx}
\usepackage{color}
\usepackage{psfrag}
\usepackage{pgfplots}
\usepackage{bm}
%%%%%%%%%%%%%%%%%%%%%

% Other Settings

%%%%%%%%%%%%%%%%%%%%%%%%%% Page Setting %%%%%%%%%%
\geometry{a4paper}

%%%%%%%%%%%%%%%%%%%%%%%%%% Define some useful colors %%%%%%%%%%%%%%%%%%%%%%%%%%
\definecolor{ocre}{RGB}{243,102,25}
\definecolor{mygray}{RGB}{243,243,244}
\definecolor{deepGreen}{RGB}{26,111,0}
\definecolor{shallowGreen}{RGB}{235,255,255}
\definecolor{deepBlue}{RGB}{61,124,222}
\definecolor{shallowBlue}{RGB}{235,249,255}
%%%%%%%%%%%%%%%%%%%%%

%%%%%%%%%%%%%%%%%%%%%%%%%% Define an orangebox command %%%%%%%%%%%%%%%%%%%%%%%%
\newcommand\orangebox[1]{\fcolorbox{ocre}{mygray}{\hspace{1em}#1\hspace{1em}}}
%%%%%%%%%%%%%%%%%%%%%

%%%%%%%%%%%%%%%%%%%%%%%%%%%% English Environments 
\newtheoremstyle{mytheoremstyle}{3pt}{3pt}{\normalfont}{0cm}{\rmfamily\bfseries}{}{1em}{{\color{black}\thmname{#1}~\thmnumber{#2}}\thmnote{\,--\,#3}}
\newtheoremstyle{myproblemstyle}{3pt}{3pt}{\normalfont}{0cm}{\rmfamily\bfseries}{}{1em}{{\color{black}\thmname{#1}~\thmnumber{#2}}\thmnote{\,--\,#3}}
\theoremstyle{mytheoremstyle}
\newmdtheoremenv[linewidth=1pt,backgroundcolor=shallowGreen,linecolor=deepGreen,leftmargin=0pt,innerleftmargin=20pt,innerrightmargin=20pt,]{theorem}{Theorem}[section]
\theoremstyle{mytheoremstyle}
\newmdtheoremenv[linewidth=1pt,backgroundcolor=shallowBlue,linecolor=deepBlue,leftmargin=0pt,innerleftmargin=20pt,innerrightmargin=20pt,]{definition}{Definition}[section]
\theoremstyle{myproblemstyle}
\newmdtheoremenv[linecolor=black,leftmargin=0pt,innerleftmargin=10pt,innerrightmargin=10pt,]{problem}{Problem}[section]
%%%%%%%%%%%%%%%%%%%%%

%% Plotting Settings 
\usepgfplotslibrary{colorbrewer}
\pgfplotsset{width=8cm,compat=1.9}
%%%%%%%%%%%%%%%%%%%%%

%% Title & Author %%%
\title{SEL Activity 1}
\author{Muhammad Meesum Ali Qazalbash}
%%%%%%%%%%%%%%%%%%%%%

\begin{document}
\maketitle

\section{Activity 1a}
We know that \(X\backsim \operatorname{Ber}(p)\), and the expected/mean value of \(X\) will be \(\mu=p\). Variance is defined as,

\begin{equation}
	\begin{split}
		\operatorname{Var}[X]&:=\operatorname{E}[(X-\mu)^2]\\
		\operatorname{Var}[X]&:=\sum_{x\in X}(x-\mu)^2\operatorname{Pr}[X=x]\\
		\operatorname{Var}[X]&:=\sum_{x\in X}(x-p)^2\operatorname{Pr}[X=x]\\
		\operatorname{Var}[X]&:=(0-p)^2\operatorname{Pr}[X=0]+(1-p)^2\operatorname{Pr}[X=1]\\
		\operatorname{Var}[X]&:=p^2(1-p)+(1-p)^2p\\
		\operatorname{Var}[X]&:=p(1-p)\blacksquare\\
	\end{split}
\end{equation}

\section{Activity 1b}

A random experiment has been caried out as independent trail of Bernoulli Random Variable with same probability \(p\). The estimation for \(\hat{p}\) is given by,

\[\hat{p}=\frac{X_1+X_2+\cdots+X_n}{n}\]

The variance of \(\hat{p}\) will be,

\begin{equation}
	\begin{split}
		\operatorname{Var}[\hat{p}]&=\operatorname{Var}\left[\frac{X_1+X_2+\cdots+X_n}{n}\right]\\
		\operatorname{Var}[\hat{p}]&=\frac{1}{n^2}\operatorname{Var}\left[X_1+X_2+\cdots+X_n\right]
	\end{split}
\end{equation}
Variance of sum of independent random variable is equals to the sum of variance of those random variable.

\[\operatorname{Var}[\hat{p}]=\frac{1}{n^2}\operatorname{Var}[X_1]+\frac{1}{n^2}\operatorname{Var}[X_2]+\cdots+\frac{1}{n^2}\operatorname{Var}[X_n]\]

Variance of all \(X_i\) will be same.

\begin{equation}
	\begin{split}
		\operatorname{Var}[\hat{p}]&=\frac{1}{n^2}\operatorname{Var}[X_1]+\frac{1}{n^2}\operatorname{Var}[X_1]+\cdots+\frac{1}{n^2}\operatorname{Var}[X_1]\\
		\operatorname{Var}[\hat{p}]&=\frac{n}{n^2}\operatorname{Var}[X_1]\\
		\operatorname{Var}[\hat{p}]&=\frac{1}{n}\operatorname{Var}[X_1]\\
		\operatorname{Var}[\hat{p}]&=\frac{p(1-p)}{n}\blacksquare
	\end{split}
\end{equation}
\end{document}