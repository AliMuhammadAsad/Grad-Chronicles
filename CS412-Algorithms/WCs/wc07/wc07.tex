%%%%%%%%%%%%%%%%%%%%%%%%%%%%% Define Exam %%%%%%%%%%%%%%%%%%%%%%%%%%%%%%%%%%
\documentclass[addpoints]{exam}
%%%%%%%%%%%%%%%%%%%%%%%%%%%%%%%%%%%%%%%%%%%%%%%%%%%%%%%%%%%%%%%%%%%%%%%%%%%%%%%

%%%%%%%%%%%%%%%%%%%%%%%%%%%%% Using Packages %%%%%%%%%%%%%%%%%%%%%%%%%%%%%%%%%%
\usepackage{amsmath, amssymb, amsthm, amsfonts, geometry, venndiagram, tikz}
\usepackage{graphicx, xcolor, color, wrapfig, parskip, float, tabularx}
\usepackage[breaklinks]{hyperref}
\usepackage{colortbl, caption}
\usepackage{listings, mdframed, subfig, matlab-prettifier, hyperref, pythonhighlight}
\usepackage{lipsum, bookmark, booktabs, empheq, titlesec, verbatim, subfig, pdfpages, comment}

%%%%%%%%%%%%%%%%%%%%%%%%%%%%%%%%%%%%%%%%%%%%%%%%%%%%%%%%%%%%%%%%%%%%%%%%%%%%%%%
\definecolor{codebackground}{rgb}{0.95,0.95,0.95}
\definecolor{codegray}{rgb}{0.5,0.5,0.5}
\definecolor{codepurple}{rgb}{0.58,0,0.82}
\definecolor{codeblue}{rgb}{0.13,0.29,0.53}
\definecolor{ocre}{RGB}{243,102,25}
\definecolor{mygray}{RGB}{243,243,244}
\definecolor{deepGreen}{RGB}{26,111,0}
\definecolor{shallowGreen}{RGB}{235,255,255}
\definecolor{deepBlue}{RGB}{61,124,222}
\definecolor{shallowBlue}{RGB}{235,249,255}
\definecolor{softgray}{rgb}{0.95, 0.95, 0.95}
\definecolor{codegreen}{rgb}{0,0.6,0}
\definecolor{codegray}{rgb}{0.5,0.5,0.5}
\definecolor{codepurple}{rgb}{0.58,0,0.82}
\definecolor{backcolour}{rgb}{0.95,0.95,0.92}

%Code listing style named "mystyle"
\lstdefinestyle{mystyle}{
  backgroundcolor=\color{backcolour}, commentstyle=\color{codegreen},
  keywordstyle=\color{magenta},
  numberstyle=\tiny\color{codegray},
  stringstyle=\color{codepurple},
  basicstyle=\ttfamily\footnotesize,
  breakatwhitespace=false,         
  breaklines=true,                 
  captionpos=b,                    
  keepspaces=true,                 
  numbers=left,                    
  numbersep=5pt,                  
  showspaces=false,                
  showstringspaces=false,
  showtabs=false,                  
  tabsize=2
}

%"mystyle" code listing set
\lstset{style=mystyle}

\usetikzlibrary{arrows,shapes,positioning,shadows,trees, backgrounds}

\tikzstyle{arrow} = [->,>=stealth]
\tikzstyle{node} = [auto,font=\footnotesize,draw,circle]

%%%%%%%%%%%%%%%%%%%%%%%%%%%%% Header and Footer %%%%%%%%%%%%%%%%%%%%%%%%%%%%%%%%%%
\pagestyle{headandfoot}
\runningheadrule
\runningfootrule
\runningheader{Algorithms: Design and Analysis}{Weekly Challenge 07}{CS 412}
\runningfooter{}{Page \thepage\ of \numpages}{}
\firstpageheader{}{}{}
%%%%%%%%%%%%%%%%%%%%%%%%%%%%%%%%%%%%%%%%%%%%%%%%%%%%%%%%%%%%%%%%%%%%%%%%%%%%%%%

% Other Settings
% \boxedpoints
\printanswers
\qformat{}  %Comment this to number questions, uncomment this to not number questions

\newcommand\union\cup
\newcommand\inter\cap

%%%%%%%%%%%%%%%%%%%%%%%%%%%%%%% Title & Author %%%%%%%%%%%%%%%%%%%%%%%%%%%%%%%%

\title{Algorithms: Design and Analysis - CS 412 \vspace*{-4mm}}
\author{Weekly Challenge 07: Maximum Flows}
\date{\vspace*{-4mm} Ali Muhammad Asad - aa07190}

% \pgfplotsset{compat=1.18}

%%%%%%%%%%%%%%%%%%%%%%%%%%%%%%%%%%%%%%%%%%%%%%%%%%%%%%%%%%%%%%%%%%%%%%%%%%%%%%%

\begin{document}
\maketitle

\begin{questions}
  \question[2]
  We are going to write a class \pyth{MaxFlow} in a file \texttt{maxflow.py}. Objects of the class will take in a flow network and compute its maximum flow. Your code will be tested by \pyth{pytest} using the file, \texttt{test\_maxflow.py}, given in \texttt{WC7\_MaxFlows.zip}. To test your implementation of \pyth{MaxFlow}, open the directory containing \texttt{test\_maxflow.py} and \texttt{maxflow.py} in the terminal, and run the following command:
  \begin{lstlisting}[language=bash]
pytest test_maxflow.py
\end{lstlisting}
  \textbf{TASKS}:
  \begin{parts}
    \part The \pyth{__init__} method of \pyth{MaxFlow} takes as argument a \pyth{file} object which points to a file containing a flow network, $G$, in the following format.
    \begin{itemize}
      \item The first line contains the number of edges, $e$.
      \item Each of the next $e$ lines contains an edge in the format $u\ v\ c$ where $(u,v)$ is an edge in $G$ with capacity $c$.
      \item The source vertex is always named, $s$, and the sink vertex is always named, $t$.
      \item All capacities are integral and positive.
    \end{itemize}
    \part The \pyth{get_value} method returns the value of the maximum flow on $G$.
    \part The \pyth{get_flow} method returns the maximum flow on $G$ as a \pyth{dict} object. A key in the returned \pyth{dict} object is an edge, $(u,v)$ in the maximum flow and the value is the amount of flow along $(u,v)$. Only edges with non-zero flow are included.
    \part Ensure that all tests pass by running \pyth{pytest} locally.
    \part Do not include any external packages except \pyth{networkx} for the possible use of \href{https://networkx.org/documentation/stable/reference/classes/digraph.html}{\pyth{networkx.DiGraph}} to store and operate on the flow network.
    \part You may modify the error messages in \texttt{test\_maxflow.py} to convey more information if you wish, but you may not alter any other functionality in it.
  \end{parts}
  \newpage
  \begin{solution}
    \begin{lstlisting}
import networkx as nx
class MaxFlow:
    def __init__(self, file) -> None:
        self.dag = nx.DiGraph()
        for i in range(len(file)):
            node1, node2, capacity = file[i].split()
            self.dag.add_edge(node1, node2, capacity=int(capacity))
            if not self.dag.has_edge(node2, node1):
                self.dag.add_edge(node2, node1, capacity=0)
    
    def get_value(self):
        ''' Get the maximum flow value of the DAG '''
        residual = self.dag.copy()
        parent = {}

        max_flow = 0

        while self.bfs(residual, 's', 't', parent):
            path_flow = float("Inf")
            s = 't'
            while s != 's':
                path_flow = min(path_flow, residual[parent[s]][s]['capacity'])
                s = parent[s]
            max_flow += path_flow
            v = 't'
            while v != 's':
                u = parent[v]
                residual[u][v]['capacity'] -= path_flow
                residual[v][u]['capacity'] += path_flow
                v = parent[v]
        return max_flow

    def bfs(self, residual, source, sink, parent):
        visited = {node: False for node in residual.nodes()}
        queue = []
        queue.append(source)
        visited[source] = True

        while queue:
            u = queue.pop(0)
            for ind, val in enumerate(residual[u]):
                if visited[val] == False and residual[u][val]['capacity'] > 0:
                    queue.append(val)
                    visited[val] = True
                    parent[val] = u

        return True if visited[sink] else False

    def get_flow(self):
        ''' Get the flow of each node in the DAG '''
        residual = self.dag.copy()
        parent = {}

        flow = {}
        for u, v in self.dag.edges():
            flow[(u, v)] = 0

        while self.bfs(residual, 's', 't', parent):
            path_flow = float("Inf")
            s = 't'
            path = []
            while s != 's':
                path.append((parent[s], s))
                path_flow = min(path_flow, residual[parent[s]][s]['capacity'])
                s = parent[s]

            for u, v in path:
                flow[(u, v)] += path_flow

            v = 't'
            while v != 's':
                u = parent[v]
                residual[u][v]['capacity'] -= path_flow
                residual[v][u]['capacity'] += path_flow
                v = parent[v]

        flow = {edge: f for edge, f in flow.items() if f > 0}
        return flow
    \end{lstlisting}
  \end{solution}
\end{questions}
\end{document}