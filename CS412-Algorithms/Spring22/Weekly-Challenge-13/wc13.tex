\documentclass[a4paper]{exam}

\usepackage{amsmath}
\usepackage{geometry}
\usepackage{hyperref}

\printanswers

\title{Weekly Challenge 13: Minimum Spanning Trees (MST)\\CS 412 Algorithms: Design and Analysis}
\author{@username}  % replace with your GitHub user names
\date{Habib University\\Spring 2022}

% \qformat{{\large\bf \thequestion. \thequestiontitle}\hfill}
\boxedpoints

\begin{document}
\maketitle

\begin{questions}
  
\question[2]
We studied Prim's and Kruskal's algorithms for finding a minimum spanning tree (MST) in an undirected, weighted, connected graph. Both algorithms have the same asymptotic bounds. However, in practice, their performance depends upon the underlying data structures--a priority queue in Prim's and a disjoint set data structure in Kruskal's algorithm--and how we implement them. 

In this challenge, we will compare the performance of the two algorithms using datasets of different sizes. These are\footnote{taken from \href{https://algs4.cs.princeton.edu/43mst/}{4.3 Minimum Spanning Trees, Princeton University}}:
\begin{itemize}
\item \href{https://algs4.cs.princeton.edu/43mst/tinyEWG.txt}{tinyEWG.txt}: contains 8 vertices and 16 edges
\item \href{https://algs4.cs.princeton.edu/43mst/mediumEWG.txt}{mediumEWG.txt}: contains 250 vertices and 1,273 edges
\item \href{https://algs4.cs.princeton.edu/43mst/1000EWG.txt}{1000EWG.txt}: contains 1,000 vertices and 8,433 edges
\item \href{https://algs4.cs.princeton.edu/43mst/10000EWG.txt}{10000EWG.txt}: contains 10,000 vertices and 61,731 edges
\item \href{https://algs4.cs.princeton.edu/43mst/largeEWG.txt}{largeEWG.txt}: contains one million vertices and 7,586,063 edges
\end{itemize}

  \textbf{TASKS}:
  \begin{parts}
  \part Write a function to read the graphs from the data files linked above. You may use \href{https://networkx.org}{\texttt{networkx}}, \href{https://igraph.org/python/}{\texttt{igraph}}, any other suitable library, or an implementation of your own to store and access the graph.
    \part Implement a priority queue for use in Prim's algorithm. You may write your own or use an existing implementation, e.g. \href{https://docs.python.org/3/library/queue.html}{\texttt{queue.PriorityQueue}}.
    \part Implement a \textit{union find} or \textit{union by rank} data structure for disjoint sets. You may write your own or use an existing implementation, e.g. \href{https://python-algorithms.readthedocs.io/en/stable/python_algorithms.basic.html#module-python_algorithms.basic.union_find}{\texttt{python\_algorithms.basic.union\_find}}.
    \part Implement Prim's algorithm using an appropriate data structure from above. Do not use any library implementation of  the algorithm.
    \part Implement Kruskal's algorithm using an appropriate data structure from above. Do not use any library implementation of the algorithm.
    \part Time the execution of your implementation of Prim's algorithm on each of the graphs linked above.
    \part Time the execution of your implementation of Kruskal's algorithm on each of the graphs linked above.
    \part Compare the timings in a plot and include the plot below.
    \part Also share below any notable observations that you may have.
    \part Share your plot in a comment on the \textit{Weekly Challenge 13} post in the course group on Yammer.
  \end{parts}
 
  \begin{solution}
    % Enter your solution here.
  \end{solution}
\end{questions}
\end{document}

%%% Local Variables:
%%% mode: latex
%%% TeX-master: t
%%% End: