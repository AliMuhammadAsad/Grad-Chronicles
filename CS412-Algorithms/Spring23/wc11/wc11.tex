\documentclass[a4paper]{exam}

\usepackage[export]{adjustbox}
\usepackage{geometry}
\usepackage{graphicx}
\usepackage{hyperref}
\usepackage{mathtools}
\usepackage{tabularx}
\usepackage{titling}

\graphicspath{{images/}}

\printanswers

\title{Weekly Challenge 11: Dynamic Programming\\CS 412 Algorithms: Design and Analysis}
\author{team-name}  % <==== replace with your team name for grading
\date{Habib University | Spring 2023}

\runningheader{CS 412: Algorithms}{WC09: Dynamic Programming}{\theauthor}
\runningheadrule
\runningfootrule
\runningfooter{}{Page \thepage\ of \numpages}{}

\qformat{{\large\bf \thequestion. \thequestiontitle}\hfill}
\boxedpoints

\begin{document}
\maketitle

\begin{questions}

  
\titledquestion{Suroor-e-Milaap}

  \begin{tabularx}{\linewidth}{Xr}
    Some years from now, you are based out of station and are driving to Habib University in order to attend the alumni convention, which by now is called, ``Jashn-e-Milaap''. For reasons of respectability, you are keeping your attendance in the event a secret from your spouse and colleagues. You are driving your EV which has a range of $m$ km on a full charge. From your starting point, there are charging stations at distances $d_1 < d_2 < d_3 <\ldots<d_n$, with each station within $m$ km of the previous. At $d_n$ is also your last stop at Habib University.
    &
    \includegraphics[valign=t,scale=.7]{milaap}\\
  \end{tabularx}

  You have started your trip with a full charge and can stop at any charging station and top up your EV's charge. For a distance of $x$ km traveled between top-ups, the amount of stress to the battery is proportional to $(m-x)$.

  You want to plan your charging pit-stops so as to minimize the total stress to your EV's battery, and to pay tribute to your Algorithms course, you want to do so using the dynamic programming approach.

  \begin{parts}
  \part[3] Characterize the structure of an optimal solution.
  \part[2] Recursively define the value of an optimal solution.
  \part[3] Show how the value of an optimal solution can be computed in a bottom-up manner.
  \part[2] Argue about the time and space complexities of your approach.
  \end{parts}
  Remember to observe good attribution practices and cite any sources or references, \href{https://hulms.instructure.com/courses/2616/discussion_topics/29240}{especially if using AI}.

  \begin{solution}
    % Enter your solution here.
  \end{solution}
  
\end{questions}

\end{document}

%%% Local Variables:
%%% mode: latex
%%% TeX-master: t
%%% End:
