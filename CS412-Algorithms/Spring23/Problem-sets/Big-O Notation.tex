\documentclass{article}
\usepackage[utf8]{inputenc}
\usepackage[framemethod=tikz]{mdframed}
\usepackage{hyperref}
\usepackage{tikz}
\usetikzlibrary{arrows,automata,positioning}
\title{Practice Problems}
\author{CS 412-R1 Algorithms: Design \& Analysis}
\date{Spring 2023}

\begin{document}

\maketitle

\noindent \textbf{Properties:}
\begin{enumerate}
    \item if $f(n)=O(g(n))$ and $g(n)=O(h(n))$ then $f(n)=O(h(n))$ \\(Transitive property holds for all big-O notations $\{O,o,\theta,\Theta, \Omega\}$)
    \item if $f(n)=O(g(n))$ and $h(n)$ is non-negative then $f(n).h(n)=O(g(n).h(n))$
    \item $f(n)+g(n)=O(Max(f(n),g(n)))$
\end{enumerate}

Functions in increasing growth-rate (in $O$ sense):
$$1<\log{(n)}< \sqrt{n}< n< n\log{(n)}< n\log{^k(n)}< n^2< 2^n< n!< n^n$$

\noindent \textbf{Prove or Disprove:}
\begin{enumerate}
    \item $(n+1)^2=n^2+O(n)$
    \item $2^{n+1}=O(2^n)$
    \item $(n+O(n^{1/2})).(n+O(\log n))^2=n^3+O(n^{5/2})$
    \item $2^{(1+O(\frac{1}{n}))^2}=2+O(\frac{1}{n})$
    \item $n^{O(1)}=O(e^n)$
    \item $O(e^n)=n^{O(1)}$
    \item $n^{\log n}=O((\log n)^n)$
    \item if $f(n)=O(g(n))$ then $2^{f(n)}=O(2^{g(n)})$
    \item $2^{2n}=O(2^n)$
    \item $\sqrt{n}\log{(n)} = O(n)$
    \item $\log{(n!)} = \Theta (n \log{(n)})$
    \item $n! = o(n^n)$
    \item $n! = \omega(2^n)$
\end{enumerate}

\end{document}
