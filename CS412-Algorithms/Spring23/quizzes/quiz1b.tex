\documentclass[addpoints,a4paper]{exam}

\usepackage{amsmath, amsfonts, amssymb, amsthm}
\usepackage{array}
\usepackage{booktabs}
\usepackage{pythonhighlight}

\newcolumntype{L}{>$l<$}

% solution
\usepackage{draftwatermark}
\SetWatermarkText{Sample Solution}
\SetWatermarkScale{3}
\SetWatermarkLightness{0.9}
\printanswers

\begin{document}
\begin{flushleft}
  { \large \textsf{\textbf{CS 412: Algorithms: Design and Analysis,  Quiz 4 -- L4, Spring 2023}}}\vspace{.5em}
  
  Thursday, 26 January, 2023. Total Marks: \numpoints, Duration: 30 minutes.
\end{flushleft}

{\bf Instructions:}

\begin{enumerate}
  % \item Please observe the allowed time for this quiz as indicated on Canvas.
\item Solve the problems by hand on blank A4 paper in clear and legible handwriting.
\item Be precise and concise in your solutions.
  \item For each question, justify your answer. Simply stating the answer will not be sufficient.
\item Work in pencil will be considered rough work and ignored for grading purposes.
\end{enumerate}
\centerline{\rule{.7\textwidth}{1pt}}

\begin{questions}

  \question The running time of an algorithm is given by the following recurrence relation.
  \[
    T(n) = \Theta(n) + 2T\left(\frac{n}{3}\right)
  \]
  
  \begin{parts}
    \part[2] Write plausible pseudocode for the algorithm, identifying the lines corresponding to each of the steps: \textit{divide}, \textit{conquer}, and \textit{combine}.
    \begin{solution}
\begin{python}
def f(A, k):
  # A is a sequence of elements. k is an element from A.
  # Choose split points in A to divide into thirds.
  n = len(A)
  third = n//3
  # scan A to find the index of k
  i = A.index(k)
  # prepare portions that do not contain k,
  # prepare elements in those portions.
  if i >= split2:  # k is in the last third
    left_bounds = (0, third)
    right_bounds = (third, 2*third)
    right_index = i - third
    left_index = right_index - third
  elif i >= split1:  # k is in the middle third
    left_bounds = (0, third)
    right_bounds = (2*third, n)
    left_index = i - third
    right_index = i + third
  else:  # k is in the first third
    left_bounds = (third, 2*third)
    right_bounds = (2*third, n)
    left_index = i + third
    right_index = left_index + third
  # recurse on identified portions
  left_result = f(A[left_bound[0]: left_bound[1]], A[left_index])
  right_result = f(A[right_bound[0]: right_bound[1]], A[right_index])
  # combine the results
  return min(left_result, right_result)
\end{python}
      Divide: Lines 4 to 24\\
      Conquer: Lines 26 to 27\\
      Combine: Line 29
    \end{solution}
    \part[2] Argue how your pseudocode corresponds to the recurrence relation.
    \begin{solution}
      Let the running time of the function be $T(n)$.
      \begin{itemize}
      \item The divide step performs $\Theta(n)$ work owing to the linear scan on line 7. All the other lines for this step perform a constant amount of work.
      \item The conquer step makes 2 recursive calls on problems of size $\frac{n}{3}$. The total amount of work done is therefore $2T\left(\frac{n}{3}\right)$.
      \item The combine step takes $\Theta(1)$ time.
      \end{itemize}
      $T(n)$ is therefore $\Theta(n) + 2T\left(\frac{n}{3}\right)$.
    \end{solution}
  \end{parts}

  \question We are given the following algorithm.
\begin{python}
def f(A, n):
  for i = 1 to n-1:
    min = A[i]
    idx = i
    for j = i+1 to n:
      if A[j] < min:
        min = A[j]
        idx = j
    A[i], A[idx] = A[idx], A[i]
\end{python}
  \begin{parts}
    \part[2] State and justify an invariant for the outer loop. The invariant must capture the work done in the loop.
    \begin{solution}
      \underline{Invariant}: At the end of the $i$-th iteration, $A[i]\leq A[j]$ for all $i< j \leq n$.

      Lines 3 to 8 iterate over $i<j\leq n$ and store the running minimum and corresponding index in \texttt{min} and \texttt{idx}. Line 9 swaps the minimum with $A[i]$. Therefore, $A[i]$ ends up as the smallest element between positions $i$ and $n$.
    \end{solution}
    \part[2] Derive an expression for the run time of this algorithm. If applicable, argue about the different cases of the run time and derive corresponding expressions.
    \begin{solution}
      We can argue about each line as follows.
      
      \begin{tabular}{LLL}
        \text{Line} & \text{Cost} & \text{Number of repetitions}\\
        \hline
        2 & c_1 & n\\
        3 & c_2 & n-1\\
        4 & c_3 & n-1\\
        5 & c_4 & \sum\limits_{i=1}^{n-1} (n-i+1)\\
        6 & c_5 & \sum\limits_{i=1}^{n-1} (n-i)\\
        7 & c_6 & \sum\limits_{i=1}^{n-1} t_i\\
        8 & c_7 & \sum\limits_{i=1}^{n-1} t_i\\
        9 & c_8 & n-1\\
      \end{tabular}
      
      where $t_i$ is the number of times that lines 7 and 8 execute for a value of $i$. Grouping together lines 2, 3, 4, and 9, and lines 5, 6, 7, and 8, we get,
      \begin{align*}
        T(n) & = \Theta(n) + \sum\limits_{i=1}^{n-1} (c_4(n-i+1) + c_5(n-i) + c_6t_i + c_7t_i)\\
             & = \Theta(n) + (c_4+c_5)\sum\limits_{i=1}^{n-1}n - (c_4+c_5)\sum\limits_{i=1}^{n-1}i + c_4\sum\limits_{i=1}^{n-1}1 + (c_6+c_7)\sum\limits_{i=1}^{n-1}t_i\\
             & = \Theta(n) + (c_4+c_5)n(n-1) - (c_4+c_5)\frac{n(n-1)}{2} + c_4(n-1) + (c_6+c_7)\sum\limits_{i=1}^{n-1}t_i\\
             & = \Theta(n) + (c_4+c_5)\frac{n(n-1)}{2} + c_4(n-1) + (c_6+c_7)\sum\limits_{i=1}^{n-1}t_i\\
             & = \Theta(n) + \Theta(n^2) + (c_6+c_7)\sum\limits_{i=1}^{n-1}t_i\\
             & = \Theta(n^2) + (c_6+c_7)\sum\limits_{i=1}^{n-1}t_i\\
      \end{align*}
      \underline{Best case}: Lines 7 and 8 never execute. Then $t_i=0$, and
      \[
        T(n) = \Theta(n^2)
      \]
      \underline{Worst case}: Lines 7 and 8 execute every time line 6 executes. Then $t_i=n-i$, and
      \[
        T(n) = \Theta(n^2) + (c_6+c_7)\sum\limits_{i=1}^{n-1}(n-i) = \Theta(n^2)\\
      \]
    \end{solution}
  \end{parts}

  \question[2] Prove or disprove the following statement.
  \[
    f(n) = \Theta(g(n)) \iff g(n) = \Theta(f(n))
  \]
  \begin{solution}
    We prove both directions of the biconditional.
    \begin{proof}
      \underline{Case $f(n) = \Theta(g(n)) \implies g(n) = \Theta(f(n))$}:\\
      There exist $c_1,c_2>0$ such that $0\leq c_1g(n) \leq f(n) \leq c_2 g(n)$ for $n\geq n_0$.\\
      Then $0\leq \frac{1}{c_2}f(n) \leq g(n) \leq \frac{1}{c_1}f(n)$ for $n\geq n_0$.\\
      \smallskip
      
      \underline{Case $g(n) = \Theta(f(n)) \implies f(n) = \Theta(g(n))$}:\\
      This can be argued analogously.
    \end{proof}
  \end{solution}
\end{questions}
\centerline{\rule{.3\textwidth}{1pt}viel Gl\"uck\rule{.3\textwidth}{1pt}}

\end{document}
