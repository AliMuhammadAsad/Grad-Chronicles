\documentclass[addpoints,a4paper]{exam}
\usepackage[a4paper]{geometry}

\usepackage{amsmath, amsfonts, amssymb, amsthm}
\usepackage{tabularx}

\runningheader{CS 412 Algorithms}{Quiz 3C}{ID: \rule{.2\textwidth}{.5pt}}
\runningheadrule
\runningfootrule
\runningfooter{}{Page \thepage\ of \numpages}{}

% % solution
% \usepackage{draftwatermark}
% \SetWatermarkText{Sample Solution}
% \SetWatermarkLightness{.9}
% \SetWatermarkScale{3}
% \printanswers

\begin{document}
\begin{flushleft}
  { \large \textsf{\textbf{CS 412: Algorithms: Design and Analysis, L1, Quiz 3A, Spring 2023.}}}\vspace{.5em}
  
  \numquestions\ problems for \numpoints\ points on \numpages\ printed sides. Duration: 30 minutes. \today.
\end{flushleft}

Instructions:
\begin{enumerate}
  % \item Please observe the allowed time for this quiz as indicated on Canvas.
\item Enter your name and ID below and at the top of every side.
\item Solve the problems by hand in clear and legible handwriting in the provided space.
\item You may use the last side for rough work.
\item Provide precise and concise solutions.
\end{enumerate}

\noindent Student Name: \hrulefill \\[5pt]
\noindent Student ID: \hrulefill \\
\rule{\textwidth}{1pt}

\begin{questions}
\question[3] Explain whether a dynamic programming approach would benefit the mergesort algorithm.
\newpage
\question Consider the following variant of the rod-cutting problem. We are given a rod of length, $l$, and the price, $p_i$, for a rod of length $i$, where $1\leq i\leq l$. Our goal is to make as many cuts to the rod so as to obtain the maximum revenue. Making each cut incurs a price, $c$. The generated revenue is equal to the total price of the pieces minus the total price of the cuts.
  \begin{parts}
  \part[3] Provide a recurrence for the maximum revenue for a rod of length, $l$.
\newpage
\part[2] Show how the value of an optimal solution can be computed in a bottom-up manner.
  \vfill
\part[2] Argue about the time and space complexities of your approach.
  \vfill
  \end{parts}
\end{questions}

\newpage
\centerline{\large Rough Work}

\end{document}

%%% Local Variables:
%%% mode: latex
%%% TeX-master: t
%%% End:
