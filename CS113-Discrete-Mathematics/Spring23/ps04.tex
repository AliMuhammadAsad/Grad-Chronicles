\documentclass[addpoints]{exam}

\usepackage{amsmath}
\usepackage{amssymb}
\usepackage{geometry}
\usepackage{venndiagram}
\usepackage{graphicx}

% Header and footer.
\pagestyle{headandfoot}
\runningheadrule
\runningfootrule
\runningheader{Discrete Mathematics}{Problem Set 4}{CS/Math 113}
\runningfooter{}{Page \thepage\ of \numpages}{}
\firstpageheader{}{}{}

\boxedpoints
\printanswers
\qformat{} %Comment this to number questions, uncomment this to not number questions

\newcommand\union\cup
\newcommand\inter\cap
\newenvironment{definition}[2][Definition]{\begin{trivlist}
    \item[\hskip \labelsep {\bfseries #1}\hskip \labelsep {\bfseries #2.}]}{\end{trivlist}}
\newenvironment{problem}[2][Problem]{\begin{trivlist}
    \item[\hskip \labelsep {\bfseries #1}\hskip \labelsep {\bfseries #2.}]}{\end{trivlist}}

\title{CS/Math 113 - Problem Set 4}
\author{Dead TAs Society \\ Habib University - Spring 2023}
\date{Week 05}

\begin{document}
\maketitle
\begin{sloppypar}
\section*{Problems}
    \begin{problem}{1}
    Explain what you must do to disprove the statement:
    $x^3+5x + 3$ has a root between $x = 0$ and $x=1$
    \end{problem}
    
    \begin{questions}
        \question
        \begin{solution}

            The statement in logical notation is 
            $$\exists x \text{ such that } (0<x<1 \land x^3+5x+3 = 0)$$
            Giving a counterexample is not enough. Saying that when $x=0.5$ then $x^3 +5x+3 \neq 0$ is not sufficient. \\
            To disprove this statement, we need to prove that the \textbf{negation is true} which is
            $$\neg \exists x \text{ such that } 0<x<1 \land x^3+5x+3 = 0 \equiv \forall x \text{ such that } \neg(0<x<1 \land x^3+5x+3 = 0)$$
            Or in English
            \begin{center}
                For all $x$, it is not the case that both $x$ is between 0 and 1 and $x^3+5x+3=0$
            \end{center}

            \textbf{Alternatively:}

            \vspace*{2mm}
            For the equation to have a root between $ x = 0 \text{ and } x = 1 $, the sign must change. The change of sign indicates that there is a root, as it cuts the $x$ axis where $y$ becomes 0, therefore, the sign must change. If the sign does not change between these two values then there is no root between $ x = 0 \text{ and } x = 1 $. %[This is a simple case of a cubic equation, for higher powers, we can check for smaller intervals such as at intervals of 0.1 or 0.05.]
        \end{solution}
    \end{questions}
    \pagebreak
    \begin{problem}{2}
    Prove that for any integer $n$ the number $n^2+5n+13$ is odd
    \end{problem}
    
    \begin{questions}
        \question
        \begin{solution}
            
            \begin{parts}
                \part $n$ is even. \\ 
                If $n$ is even, then $n$ can be expressed in the form of $ n = 2k \;\;\; \forall k \in \mathbb{Z} $. \\ 
                Then, $ (2k)^2 + 5(2k) + 13 \implies 4k^2 + 10k + 13 \implies 2(2k^2 + 5k) + 13 $ \\ 
                It is clear that $ 2(2k^2 + 5k) $ is of the form $2k$, therefore it is an even number and $ 2k^2 + 5k = \alpha$. Then we get $ 2\alpha + 13 $. Since $ 2\alpha $ is even, and 13 is odd, therefore the result must be odd as the sum of even and odd is always odd. [Proved previously]
            
                \part $n$ is odd. \\ 
                $n$ can be written in the form $ n = 2k + 1 \;\;\; \forall k \in \mathbb{Z}$. \\ 
                Then, $ (2k+1)^2 + 5(2k+1) + 13 \implies 4k^2 + 4k + 1 + 10k + 5 + 13 \implies 4k^2 + 14k + 19 \implies 2(2k^2 + 7k) + 19 \implies 2\alpha + 19 $. Since $2\alpha$ is even, as it is of the form $2k$, and 19 is odd, then the result is odd, since the sum of even and odd is odd [proved previously in the course].
            \end{parts}
            Therefore, $ n^2 + 5n + 13 $ is odd $ \forall n \in \mathbb{Z} $.
        \end{solution}
    \end{questions}

    \begin{problem}{3}
    State the statement of Contradiction and verify that it is a valid argument.\\
    \textbf{Hint:} In contradiction we are saying that $A$ implies $B$ is the same as saying that $A$ and $\neg B$ happening together is false.
    \end{problem}

    \begin{questions}
        \question
        \begin{solution}
            
            For $ p \implies q $, the Contrapositive is $ \neg q \implies \neg p $.
            
            Statement is 
            \[(A \implies B ) \equiv ((A \land \neg B) \;\; is \;\; \text{false})\]
            We can show that one side is equivalent to the other
            \begin{equation*}
                \neg (A \land \neg B) \equiv (\neg A \lor B) \equiv (A \implies B)
            \end{equation*}
            Therefore it is true

        \end{solution}
    \end{questions}
    \pagebreak
    \begin{problem}{4}
    Show through contraposition the following proposition is true: $x \in \mathbb{Z}$. If $7x + 9$ is even, then $x$ is odd.
    \end{problem}

    \begin{questions}
        \question
        \begin{solution}
            
            For $ p \implies q $, the Contrapositive is $ \neg q \implies \neg p $.

            Let $ p $ be `7$x$ + 9  is even'. \\ Let $q$ be `$x$ is odd'.

            Then the contrapositive becomes, $ \neg q\implies \neg p $. If $x$ is even, then $ 7x + 9 $ is odd. 

            If $x$ is even, then $x$ can be represented in the form $x = 2k$ where $ k \in \mathbb{Z} $. 
            
            Then $ 7x + 9 = 7(2k) + 9 \implies 2(7k) + 9 \implies 2(\alpha) + 9 $ where $ 2\alpha $ is an even integer. 
            
            Since 9 is odd, and the sum of even and odd is odd, therefore, the sum is odd. Hence proved by Contrapositive.
        \end{solution}
    \end{questions}

    \begin{problem}{5}
    Prove that ``$(a+b)^2 = a^2 +b^2$'' is \textbf{not} an algebraic identity where $a,b \in \mathbb{R}$
    \end{problem}

    \begin{questions}
        \question
        \begin{solution}
            
            $ (a + b)^2 = a^2 + b^2 $ \\ Working on LHS: 
            $ \implies (a + b)(a + b) $ \\ $ \implies a^2 + ab + ba + b^2 $ \\ $ \implies a^2 + ab + ab + b^2 $ [since $ab$  and $ba$ commute for real numbers] \\ $ \implies a^2 + 2ab + b^2 \neq a^2 + b^2 $

            Hence is \textbf{not} an algebraic identity where $ a, b \in \mathbb{R} $

            \underline{The above is a common mistake.} 
            
            How do you know $ a^2 + 2ab + b^2 $ is not the same as $ a^2 + b^2 $? It is no answer to say they look different - after all, $ \sin^2\theta + \cos^2\theta $ is equal to 1 even though they look very different from 1. But it is still an indentity. Moreover, $ a^2 + 2ab + b^2 $ is the same as $ a^2 + b^2 $ for various values, such as $ a = 2 $ and $ b = 0 $. Both give the same result. $ a^2 + b^2 = 2^2 + 0^2 = 4 + 0 = 4 $ and $ (a+b)^2 = (2 + 0)^2 = (2)^2 = 4 $. 

            Therefore they give the same result. 

            \vspace*{5mm}
            We can show that they are not the same by finding any simple counterexample. A simple counterexample that can be considered is $ a = 2 $ and $ b = 1 $. Then $$ (a + b)^2 = (2 + 1)^2 = 3^2 = 9$$ while $$a^2 + b^2 = 2^2 + 1^2 = 4 + 1 = 5 $$
            And clearly $ 9 \neq 5 $. Hence the statement is not an identity.
        \end{solution}
    \end{questions}

    \begin{problem}{6}
    Prove that for $m$ and $n$ integers, if 2 divides $m$ or 10 divides $n$, then 4 divides $m^{3}n^{2}$
    \end{problem}

    \begin{questions}
        \question
        \begin{solution}
            
            For $m$ and $n$, if 2 divides $m$, then $m$ is a factor of 2, and can be represented as $ m = 2k $. If 10 divides $n$, then $n$ is a factor of 10 and can be represented as $ n = 10l $. \\ 
            Then $m^3 = 2k * 2k * 2k = 2.2.2(k^3) = 8k^3 = 4(2k^3)$. \\ 
            And $ n^2 = 10l * 10l = 10.10(l^2) = 100l^2 = 4(25l^2) $.

            If either 2 or 10 divides $m$ or $n$ respectively, then $m^3n^2$ must be divisible by 4, both can be expressed as a factor of 4. Hence proved. 
        \end{solution}
    \end{questions}

    \begin{problem}{7}
    Give a counterexample to the statement
    \begin{center}
        ``If $n$ is an integer and $n^2$ is divisible by 4, then $n$ is divisible by 4''
    \end{center}
    \end{problem}

    \begin{questions}
        \question
        \begin{solution}
            
            The simplest counterexample can be taken as $n = 2$. Then $n^2 = 4$ which is divisible by 4, however, 2 is not. Hence disproved by counterexample.
        \end{solution}
    \end{questions}

    \begin{problem}{8}
    Show through contraposition the following proposition is true : If $x^{2} - 6x + 5$ is even, then x is odd.
    \end{problem}

    \begin{questions}
        \question
        \begin{solution}
            
            Let $p$ be `$x^2 - 6x + 5$ is even' \\ 
            Let $q$ be `$x$ is odd'

            Then by the statement $ p \implies q $ \\ 
            By Contrapositive, $ \neg q \implies \neg p $ \\ 
            Therefore, if $x$ is even, $ x^2 -6x + 5 $ is odd. \\ 
            Then $x$ can be represented as $ x = 2k $ where $ k \in \mathbb{Z} $. \\ Then $x^2 - 6x + 5 = (2k)^2 - 6(2k) + 5 \\ 
            \implies 4k^2 -12k + 5 \\ \implies 2(2k^2 - 6k) + 5 \implies 2\alpha + 5$ where $ \alpha = 2k^2 - 6k $. Then $ 2\alpha $ is also an even number as it is of the form $ 2k $, and since the sum of an even number and an odd number is odd (5 is odd), therefore, the result is odd. 
            
            Hence proved by contrapositive.
        \end{solution}
    \end{questions}
    \pagebreak
    \begin{problem}{9}
    Show that any composite three-digit number must have a prime factor less than or equal to 31.
    \end{problem}

    \begin{questions}
        \question
        \begin{solution}
            
            The largest composite three-digit number is 999.
            
            Any composite number can be expressed as the product of two or more primes. For a three-digit composite number, the primes must be at most the square root of the number. Therefore, to show that any composite three-digit number must have a prime factor less than or equal to 31, it suffices to show that there are no primes greater than 31 that divide any of the numbers from 100 to 999.

            Then $ 31^2 = 961 $ which is less than 999. Henceforth it suffices. 

            Then any three digit number that is divisible by a prime greater than 31 would have to be divisible by the next prime number that is 37. $ 37^2 = 1369 $ which is a 4-digit composite number. Hence shown.
        \end{solution}
    \end{questions}

    \begin{problem}{10}
    Show that if $a$ is a positive integer and $\sqrt[n]{a}$ is rational, then $\sqrt[n]{a}$ must be an integer.
    \end{problem}

    \begin{questions}
        \question
        \begin{solution}
            
        Let $a \in \mathbb{Z}^+$, suppose $\sqrt[n]{a}$ is rational, we show that then $\sqrt[n]{a}$ must be an interger.
        \\Let $\sqrt[n]{a} = \frac{p}{q}$, where $p,q \in \mathbb{Z}$ where $q \neq 0$ and $\text{gcd}(p,q) = 1$.
        $$\sqrt[n]{a} = \frac{p}{q} \Leftrightarrow a = \frac{p^n}{q^n} \Leftrightarrow a q^n = p^n$$
        Now we have that $q^n | p^n$, but as $\text{gcd}(p,q) = 1$ then $\text{gcd}(p^n,q^n) = 1$.
        \\So as only common divider of $p^n$ and $q^n$ is 1 and $q^n | p^n$ then $q^n = 1$
        \\Therefore $ a = \frac{p^n}{q^n} = p^n$, so $\sqrt[n]{a} = p$.
        \\Which means $\sqrt[n]{a}$ is an integer.
        \end{solution}
    \end{questions}
    \pagebreak
    \begin{problem}{11}
    Prove the following claim: There exists irrational numbers $a$ and $b$ such that $a^b$ is rational.
    \end{problem}

    \begin{questions}
        \question
        \begin{solution}
            
            It is enough to prove this claim through an example that there exists rational numbers $a$ and $b$ such that $a^b$ is raional. 

            Consider $a = \sqrt{2}$ and $ b = \sqrt{2} $. Then a number $ c = a^b = \sqrt{2}^{\sqrt{2}} $ where $ c \in \mathbb{R} $

            Then either $ c $ is rational, or $c$ is irrational. 

            \textbf{Case 1:} $c$ is rational. \\ 
            If $ c = \sqrt{2}^{\sqrt{2}} $ is rational, then we already have our irrational numbers $a$ and $b$ such that $ a^b $ is rational.
            
            \textbf{Case 2:} $c$ is irrational. \\ 
            If $c = \sqrt{2}^{\sqrt{2}} $ is irrational, then let $ a = \sqrt{2}^{\sqrt{2}} $ and $ b = \sqrt{2} $. Then $$ c = \biggl( \sqrt{2}^{\sqrt{2}} \biggr)^{\sqrt{2}} = 2$$
            and 2 is rational.

            Hence proved that there exists irrational numbers $a$ and $b$ such that $ a^b $ is rational.
        \end{solution}
    \end{questions}

    \begin{problem}{12}
    Show that $\sqrt{2}$ is irrational. In other words, $\sqrt{2}$ cannot be written in the form $\frac{p}{q}$ where $p,q \in \mathbb{Z}$ and $q \neq 0$
    \end{problem}
    
    \begin{questions}
        \question
        \begin{solution}
            
            Assume $\sqrt{2}$ is rational, then $\sqrt{2} = \frac{p}{q}$, where $p,q \in \mathbb{Z}$ and $q \neq 0$.
            \\And $\frac{p}{q}$ is the lowest form it can be. 
            $$\left(\frac{p}{q}\right)^2 = 2$$
            $$p^2 = 2 q^2$$
            This implies $p$ is even which means $p = 2k$, for some $k \in \mathbb{Z}$
            $$4k^2 = 2 q^2$$
            $$2k^2 = q^2$$
            This implies $q$ is even.
            \\But $p$ and $q$ can't both be even as they are in the lowest form possible thus the 2 would be canceled. Hence we have a contradiction.
            \\Thus $\sqrt{2}$ cannot be written in form $\frac{p}{q}$ where $p,q \in \mathbb{Z}$
            \\Thus $\sqrt{2}$ is irrational. Hence proved by contradiction.
        \end{solution}
    \end{questions}

    \begin{problem}{13}
    Given that $p$ is a prime and $p|a^n$, prove that $p^n|a^n$. 
    \end{problem}

    \begin{questions}
        \question
        \begin{solution}
            
            As $p|a^n$ then $a^n = kp$ for some integer $k$.
            \\\textbf{Case 1:} $p \neq a$
            \\Then $a$ is not a prime, then $a = p_1\times p_2 \times ... p_m$
            \\$a^n = p_1^n\times p_2^n \times ... p_m^n = kp$
            \\As $p|a^n$ and $a^n = p_1^n\times p_2^n \times ... p_m^n$ then there must be some $p_i$ from $1\leq i \leq m$ such that $p|p_i$
            \\As $p_i$ is prime for all $i \leq i \leq m$, then if $p|p_i$ then $p_i = p$ which means $p|a$
            \\Then $a = pq$ so $a^n = p^n q^n$ therefore $p^n|a^n$.
            \\\textbf{Case 2:} $p = a$ 
            \\If $p = a$  and $p|a^n$ then as $a^n|a^n$ and $a^n=p^n$ then $p^n|a^n$.
        \end{solution}
    \end{questions}

    \begin{problem}{14}
    Show that there are infinitely many primes, in other words the set containing all prime numbers is infinite.
    \end{problem}

    \begin{questions}
        \question
        \begin{solution}
            
            \textbf{Definition:} A prime number is a Natural number that is only divisible by 1 and itself, and has to be divisible by 2 different numbers.
            \\\textbf{Fundamental Theorem of Arithmetic:} Every integer $N > 1$ has a prime factorization, meaning either $N$ is itself prime or can be written as a product of prime numbers.

            Let $s=\{p_0,p_1,p_2,...,p_n\}$ be set of all primes. 
            \\Let $P = p_0 \times p_1 \times p_2 \times ... \times p_n$
            \\Let $q = P+1$
            \\\textbf{Case 1:}
            \\$q$ is prime, which is not in our set $s$
            \\\textbf{Case 2:} 
            \\if $q$ is not prime, then there exits a prime factor decomposition of $q$.
            \\Let $f$ be a prime that divides $q$, then $f$ would be in our set $s$ thus $f$ would divide $P$ too. 
            \\As $f$ divides $q$ and $P$ then $f$ divides $q-P$, which is $1$
            \\Then $f$ divides 1.
            \\As $f\geq2$ $f$ cannot divide 1, thus we have a contradiction.

            Hence, the new prime number $q$ does not exist in our set, but lies outside the set. But we claimed that our set contains all prime numbers. Therefore, there are infinite prime numbers.
            
            Hence proved by contradiction.
        \end{solution}
    \end{questions}

\end{sloppypar}
\end{document}