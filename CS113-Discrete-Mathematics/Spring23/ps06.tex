\documentclass[addpoints]{exam}

\usepackage{amsmath}
\usepackage{amsthm}
\usepackage{amssymb}
\usepackage{geometry}
\usepackage{venndiagram}
\usepackage{graphicx}
\usepackage{multicol}
\usepackage{multirow}
\usepackage{array}
\usepackage{geometry}
\usepackage[shortlabels]{enumitem}

% Header and footer.
\pagestyle{headandfoot}
\runningheadrule
\runningfootrule
\runningheader{Discrete Mathematics}{Problem Set 6}{CS/Math 113}
\runningfooter{}{Page \thepage\ of \numpages}{}
\firstpageheader{}{}{}

\boxedpoints
\printanswers
\qformat{} %Comment this to number questions, uncomment this to not number questions

\newcommand\union\cup
\newcommand\inter\cap
\renewcommand{\questionlabel}{Question \thequestion.}
\renewcommand{\questionshook}{\leftmargin=0pt%
  \labelwidth=-\labelsep}

\title{CS/Math 113 - Problem Set 6}
\author{Dead TAs Society \\ Habib University - Spring 2023}
\date{Week 07}

\begin{document}
\maketitle
\begin{sloppypar}
\section*{Problems}
\begin{questions}
    \question\textbf{Problem 1.}[Chapter 2.3, Question 1]
    Why is $f$ not a function from $\mathbb{R}$ to $\mathbb{R}$ if 
    \begin{enumerate}[(a)]
        \item $ f(x) = \frac{1}{x}$
        \item $ f(x) = \sqrt{x}$
        \item $ f(x) = \pm  \sqrt{(x^2 + 1)}$
    \end{enumerate}
        \begin{solution}
            \begin{enumerate}[(a)]
                \item $f(x)$ is not defined for $x = 0$, therefore, is not a function.
                \item $f(x)$ is not defined for negative values of $x$.
                \item By the definition, a function from $A$ to $B$ is an assignment of exactly one element of $B$ to each element of $A$. However, in this case, we have two values for every $x$; the positive and negative root of $ x^2 + 1 $. Therefore, it is not a function.
            \end{enumerate}
        \end{solution}
% \pagebreak
    \question\textbf{Problem 2.}[Chapter 2.3, Question 12, 13]
Determine whether each of these functions from $\mathbb{Z}$ to $\mathbb{Z}$ is one to one. Determine which functions are onto ?
\begin{enumerate}[(a)]
    \item $f(n) = n - 1$
    \item $f(n) = n^2 + 1$
    \item $f(n) = n^3 $
    \item $f(n) = \lceil \frac{n}{2} \rceil $
\end{enumerate}
        \begin{solution}
            \begin{enumerate}[(a)]
                \item This is one-to-one as each assignment of distinct values would give a distinct result. $ n_1 -1 = n_2 -1 \implies n_1 = n_2 $. Therefore, it is also onto.
                \item It is not one-to-one as we can get the same result for different values of $x$. $f(2) = 5$ and $ f(-2) = 5 $. It is not onto function as negative integers are not in the range of the function.
                \item It is a one-to-one function as each assignemnt of distinct values would give a distinct result. $ n^3_1 = n^3_2 \implies n_1 = n_2 $. However, it is not an onto function as 3 is not the cube of any integer, therefore, there does not exist any such assignment for integer values.
                \item It is not a one-to-one function as different assignments can give the same result. Consider, $ f(1) = \lceil \frac{1}{2} \rceil = 1$ and $ f(2) = \lceil \frac{2}{2} \rceil = 1$. Therfore, is not a one-to-one function. This function is onto since an assignment can be made from each integer to any other integer. $ f(2x) = \lceil \frac{2x}{2} \rceil = \lceil x \rceil = x \hspace*{5mm} \forall x \in \mathbb{Z} $.  
            \end{enumerate}
        \end{solution}
% \pagebreak
    \question\textbf{Problem 3.}[Chapter 2.3, Question 22]
    Determine whether each of these functions is a bijection from $\mathbb{R}$ to $\mathbb{R}$.
        \begin{enumerate}[(a)]
            \item $f(x) = -3x + 4$
            \item $f(x) = -3x^2 + 7$
            \item $f(x) = \frac{(x+1)}{(x+2)} $
            \item $f(x) = x^5 + 1 $
    \end{enumerate}
        \begin{solution}
            \begin{enumerate}[(a)]
                \item The function is a one-to-one function as each unique element would have a unique result. $ f(a) = f(b) \implies a = b $. Moreover, it is an onto function, since the inverse exists; $ f^{-1}(x) = \frac{4-x}{3} $ and is clearly a one-to-one function by the definition. Therefore, $ f(x) $ is also an onto function which implies it is a bijection.
                \item It is not a one-to-one function, $ f(1) = 4 $ and $ f(-1) = 4 $. Hence is not a bijection.
                \item It is not a bijection since the function is not defined for $ x = -2 $, therefore, is not a one-to-one function. Therefore is not a bijection. 
                
                [It is interesting to note that the function can be made a bijection by modifying the domain from $ \mathbb{R} $ to $\mathbb{R}$, and changing it to $ \mathbb{R} - \{-2\} $ to $ \mathbb{R} - \{1\} $ as the inverse is $ f^{-1}(x) = \frac{(1 - 2x)}{x - 1} $]

                \item It is a one-to-one function, as each unique element would have a unique result. $ f(a) = f(b) \implies a = b $. Moreover, it is an onto function as the inverse, $ f^{-1}(x) = \sqrt[5]{x - 1} $ is also defined for all values.
            \end{enumerate}
        \end{solution}

    \question\textbf{Problem 4.}[Chapter 2.3, Question 24]
    Let $ f: \mathbb{R} \rightarrow \mathbb{R}$ and let $ f(x) > 0 $ for all $ x \in \mathbb{R}$. Show that $f(x)$ is strictly increasing if and only if the function $g(x) = \frac{1}{f(x)}$ is strictly decreasing. 
        \begin{solution}
            Suppose that $ f(x) $ is strictly increasing. Then by the definition, a function $ f(x) $ if $ f(x) > f(y) $ whenever $ x > y $. So if $ f(x) $ is strictly increasing, then we know that $ f(x) > f(y) $ where $ x > y $. Then $ g(x) < g(y) $ when $ x > y $ since the larger the denominator, the smaller the value. And this follows the definition of a strictly decresing function. Therefore, if $f(x)$ is stricty increasing, then $g(x)$ is striclty decreasing where $ g(x) = \frac{1}{f(x)} $. 

            Suppose that $ g(x) $ is strictly decreasing. Then by the definition, $ g(x) > g(y) $ where $ x < y $. Then $ f(x) < f(y) $ when $ x < y $ as $ f(x) = \frac{1}{g(x)} $. Therefore, it follows the definition of a strictly increasing function. 

            Hence proved. 
        \end{solution}

    \question\textbf{Problem 5.}[Chapter 2.3, Question 26]
    \begin{enumerate}[(a)]
        \item Prove that a strictly increasing function from $\mathbb{R}$ to itself is one to one.
        \item Give an example of an increasing function from $\mathbb{R}$ to itself is not one to one.
    \end{enumerate}
        \begin{solution}
            \begin{enumerate}[(a)]
                \item By the definition, a function $ f(x) $ is strictly increasing if $ f(x) > f(y) $ whenever $ x > y $. Then $ f(a) = f(b) \implies a = b$ as it would else violate the definition of the function being strictly increasing. Hence, it is also a one-to-one function.
                \item $ f(x) = \lceil x \rceil $ is incresing but not one-to-one. [Any constant function can also be considered as it would be both increasing and decreasing such as $ y = k \; | k \in \mathbb{R}$]
            \end{enumerate}
        \end{solution}

    \question\textbf{Problem 6.}[Chapter 2.3, Question 29]
    Show that the function $f(x) = \lvert x \rvert$ from the set of real numbers to the set of nonnegative real numbers is not invertible, but if the domain
    is restricted to the set of nonnegative real numbers, the resulting function is invertible.
        \begin{solution}
            The function $ f(x) = | x | $ is not one-to-one as the same result can be obtained by different elements. $ f(1) = f(-1) = 1 $. Hence it is not one-to-one. 
            
            By restricting the domain, the function becomes $ f(x) = x $, which is a one-to-one function and an onto function. So it is invertible and its inverse is also defined as $ f^{-1}(x) = x $ [The function is its own inverse].
        \end{solution}
\pagebreak
    \question\textbf{Problem 7.}[Chapter 2.3, Question 33]
    Suppose that $g$ is a function from A to B and $f$ is a function from B to C.
    \begin{enumerate}[(a)]
        \item Show that if both $f$ and $g$ are one-to-one functions,
        then $f \circ g$ is also one-to-one.
        \item Show that if both $f$ and $g$ are onto functions, then $f \circ g$
        is also onto.
    \end{enumerate}
        \begin{solution}
            \begin{enumerate}[(a)]
                \item By definition, if a function $f(x)$ is one-to-one, then $ f(a) \neq f(b) $ if $ a \neq b $. Then if $ f $ and $g$ are both one-to-one function, we know for two elements $x$ and $y$ such that $ x \neq y $, then $ g(x) \neq g(y) $. Since $ g(x) \neq g(y) $, and $f$ is also one-to-one, we can conclude that $ f(g(x)) \neq f(g(y)) $ where $ x \neq y $. \\ Hence shown.
                \item By definition, if a function $f(x)$, is onto, then for any element $ b \in B$, $ \exists a \in A \ni f(a) = b $. Then for any $ c \in C, \exists a \in A \ni f(g(a)) = c $. We know that $f$ is onto, therfore, for any element $c \in C$, there exists some element $b \in B$ that maps onto $c$. Similarly, since $g$ is onto, then for any element $ b \in B $, there exists some element $ a \in A $ that maps onto $b$. Therefore, we can conclude that for $ c \in C, \exists a \in A \ni f(g(a)) = c $. Hence shown.
            \end{enumerate}
        \end{solution}

    \question\textbf{Problem 8.}[Chapter 2.3, Question 34]
    Suppose that $g$ is a function from A to B and $f$ is a function from B to C. Prove each of these statements
        \begin{enumerate}[(a)]
            \item If $f \circ g$ is onto, then $f$ must also be onto.
            \item If $f \circ g$ is one-to-one, then $g$ must also be one-to-one.
            \item If $f \circ g$ is a bijection, then $g$ is onto if and only if $f$ is one to one. 
        \end{enumerate}
        \begin{solution}
            \begin{enumerate}[(a)]
                \item Suppose that $ f \circ g $ is onto. Then let $c$ be an arbitrary element such that $ c \in C $. Since $f \circ g$ is onto, then $ \exists a \in A $ such that $ (f \circ g) (a) = c \implies f(g(a)) = c$. Then let $b$ be any arbitrary element such that $b \in B$. Then $ f(b) = c \implies g(a) = b $. Therefore, for any $c \in C$, $ \exists b \in B $ such that $ f(b) = c $. This follows from the definition of onto since for every element in $C$, there is some element that can be mapped onto an element in $C$. Hence $f$ must also be onto, if $f \circ g$ is onto.
                
                \item Suppose that $ f \circ g $ is one-to-one. Consider any two arbitrary elements $ a_1, a_2 $ such that $ a_1, a_2 \in A $. Suppose that $ g(a_1) = g(a_2)$. Then $ f \circ g (a_1) = f \circ g (a_2) \implies f(g(a_1)) = f(g(a_2))$. We know that $ f \circ g $ is one-to-one. Then by the definition, if $ f \circ g (x) = f \circ g (y) $, then $ x = y $. Then following from the definition, $ a_1 = a_2 $. Since $ g(a_1) = g(a_2) $ and $ a_1 = a_2 $, then $g$ is also one-to-one.   
                
                \item Suppose that $ f \circ g $ is a bijection, then it must be one-to-one and onto. Consider $ b_1, b_2 \in B $ where $ b_1, b_2 $ are any two arbitrary elements such that $ f(b_1) = f(b_2) $. 
                
                Since $g$ is onto, then by the definition of onto functions, $ \exists a_1, a_2 \in A $ such that $ g(a_1) = b_1 $ and $ g(a_2) = b_2 $. Then $ f \circ g (a_1) = f \circ g (a_2) = f(g(a_1)) = f(g(a_2)) = f(b_1) = f(b_2) $. Since $ f \circ g $ is a bijection, it is one-to-one, hence, $a_1 = a_2 \implies g(a_1) = g(a_2)$. Then $ \forall b_1, b_2 \in B $ such that $ f(b_1) = f(b_2) $, $ b_1 = b_2 $. Therefore, by the definition, $f$ is one-to-one if $g$ is onto. 
                
                Now suppose that $f$ is one-to-one. Let $ b \in B $. Since $ f \circ g $ is a bijection, we know it is onto, so $ \exists a \in A $ such that $ f(b) = f(g(a)) = f \circ g (a) $. Since $f$ is one-to-one, $ g(a) = b $. Then $ \forall b \in B, \exists a \in A $ such that $ g(a) = b $. Which follows the definition of an onto function. Therefore $g$ is onto. 
                
                Hence proved. 
            \end{enumerate}
        \end{solution}
% \pagebreak
    \question\textbf{Problem 9.}[Chapter 2.3, Question 72]
    Suppose that $f$ is a function from A to B, where A and B
    are finite sets with $ \lvert A \rvert =  \lvert B \rvert$. Show that $f$ is one-to-one if
    and only if it is onto.
        \begin{solution}
            Suppose $f$ is one-to-one. Then every element $a \in A$, is mapped onto an element $b \in B$. We know that the cardinality of $A$ is equal to the cardinality of $B$. $ |A| = |B| $. Therefore, the number of elements in $A$ is equal to the number of elements in $B$. Hence, if every element of $A$ is mapped onto an element in $B$, and $|A| = |B|$, then we can conclude that every element of $B$ has been mapped onto by every element in $A$. Therefore, $f$ is also an onto function. 

            Suppose that $f$ is an onto function, then every element of $B$ is mapped onto by some element in $A$. We know that cardinality of $B$ is equal to the cardinality of $A$. $|A| = |B|$. Then the number of elements in $B$ is equal to the number of elements in $A$. So if $B$ is onto, then it has a unique mapping by each element from $A$, therefore, every element of $A$ maps onto some element in $B$ which implies that $f$ is also one-to-one.

            Hence proved.
        \end{solution}
\end{questions}
\end{sloppypar}
\end{document}

