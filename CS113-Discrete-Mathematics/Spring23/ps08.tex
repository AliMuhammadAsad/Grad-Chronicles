\documentclass[addpoints]{exam}

\usepackage{amsmath}
\usepackage{amsthm}
\usepackage{amssymb}
\usepackage{geometry}
\usepackage{venndiagram}
\usepackage{graphicx}
\usepackage{multicol}
\usepackage{multirow}
\usepackage{array}
\usepackage{geometry}
\usepackage{url}
\usepackage[shortlabels]{enumitem}

% Header and footer.
\pagestyle{headandfoot}
\runningheadrule
\runningfootrule
\runningheader{Discrete Mathematics}{Problem Set 8}{CS/Math 113}
\runningfooter{}{Page \thepage\ of \numpages}{}
\firstpageheader{}{}{}

\boxedpoints
\printanswers
\qformat{} %Comment this to number questions, uncomment this to not number questions

\newcommand\union\cup
\newcommand\inter\cap
\newcommand{\N}{\mathbb{N}}
\newcommand{\Z}{\mathbb{Z}}
\newcommand{\olsi}[1]{\,\overline{\!{#1}}} % overline short italic

\title{CS/Math 113 - Problem Set 8}
\author{Dead TAs Society \\ Habib University - Spring 2023}
\date{Week 12}

\begin{document}
\maketitle
\begin{sloppypar}
\section*{Problems}
\begin{questions}
\question

\textbf{Problem 1. [Chapter 2.5, Question 1]}
Determine whether each of these sets is finite, countably
infinite, or uncountable. For those that are countably infinite, exhibit a one-to-one correspondence between the
set of positive integers and that set.
\begin{enumerate}[(a)]
    \item the negative integers
    \item the even integers
    \item the integers less than 100
    \item the real numbers between 0 and $\frac{1}{2}$
    \item the positive integers less than 1,000,000,000
    \item the integers that are multiples of 7
\end{enumerate}
\begin{solution}
    \begin{parts}
        \part Countably infinite, as a one-to-one correspondence can be made by mapping each negative integer to its positive or absolute value: $ 1 \leftrightarrow -1, 2 \leftrightarrow -2, 3 \leftrightarrow -3, \dots $
        
        \part Countably infinite, as a one-to-one correspondence can be made by first reordering the even integers like so; $ 0, -2, 2, -4, 4, -6, 6, ... $ and then mapping each even integer with the set of positive integers. 

        \part Countably infinite, as a one-to-one correspondence can be made by ordering the elements in descending order; $ 99, 98, 97, 96, \dots $ and then mapping each integer with the set of positive integers.
        
        \part Uncountable. A subset of an interval of real numbers is uncountable because there is always a real number in between any two real numbers.

        \part It is a finite set ranging from 1 to 999,999,999. 

        \part Countably infinite by the same logic as in part (b). We can make a sequence: $ 0, -7, 7, -14, 14, \dots $ and make a mapping to the positive integers in the natural order. 
    \end{parts}
\end{solution}

\question
\textbf{Problem 2. [Chapter 2.5, Questions 10]}
Given an example of two uncountable sets $A$ and $B$ such that $ A - B$ is
\begin{enumerate}[(a)]
    \item finite
    \item countably infinite
    \item uncountable
\end{enumerate}
\begin{solution}
    \begin{parts}
        \part Let $A$ and $B$ both be the set of real numbers. Then $ A - B = $ \O which is finite

        \part Let $A$ be the set of real numbers and $B$ be the set of real numbers that are not integers. Then $ B = A - \mathbb{Z}$. Then $ A - B = \mathbb{Z} $ which is countably infinite.

        \part Let $A$ be the set of real numbers and $B$ be the set of real numbers except numbers between 0 and 1. Then $ A - B $ gives us the set of real numbers between 0 and 1 which is uncountable.
    \end{parts}
\end{solution}

\question
\textbf{Problem 3. [Chapter 2.5, Questions 11]}
Given an example of two uncountable sets $A$ and $B$ such that $ A \cap B$ is
\begin{enumerate}[(a)]
    \item finite
    \item countably infinite
    \item uncountable
\end{enumerate}
\begin{solution}
    \begin{parts}
        \part Let $A$ be the interval $ (0, 1) $ and $B$ be the interval $ (2, 3) $. They are both uncountable, however, $ A \cap B = $ \O \; which is finite.
        
        \part We need to find some common elements in between the two uncountable sets such that their intersection remains countable. Then let $A$ be set in part (a) adjoined with the set of integers; $ A = (0, 1) \cup \mathbb{Z} $ and $B$ be the set in part (a) adjoined with the set of integers; $ B = (2, 3) \cup \mathbb{Z} $. Then $ A \cap B = \mathbb{Z} $ which is countably infinite.

        \part Just find any interval that overlaps, then their intersection will also be uncountabe. Let $A$ be the interval $ (0, 1) $ and $B$ be the interval $ (0, 2) $. Then $ A \cap B = (0, 1) $ which is uncountable.
    \end{parts}
\end{solution}
\pagebreak
\question
\textbf{Problem 4. [Chapter 2.5, Questions 12]}
Show that if $A$ and $B$ are sets and $ A \subset B $ then $ \mid A \mid \leq \mid B \mid $
\begin{solution}
    If $ A \subset B $, then we can establish a one-to-one function from $A$ to $B$ such that each element in $A$ is mapped onto an element in $B$; $ f : A \rightarrow B $. Since a one-to-one function can be made from $A$ to $B$, this fits the definiton of $ |A| \leq |B| $. Hence shown.    
\end{solution}

\question
\textbf{Problem 5. [Chapter 2.5, Questions 14]}
    Show that if $A$ and $B$ are sets with the same cardinality, then $ \mid A \mid \leq \mid B \mid$  and $ \mid B \mid \leq \mid A \mid  $
\begin{solution}
    If $A$ and $B$ have the same cardinality, then there exists a one-to-one function from $A$ to $B$ such that $ f : A \rightarrow B $. Then $f$ meets the definiton of $ |A| \leq |B| $. 

    Now consider $ f^{-1} $ such that $ f^{-1} : B \rightarrow A $. Since the cardinalities of $A$ and $B$ are the same, then there also exists a one-to-one function from $B$ to $A$ defined by $ f^{-1} $. Then $ f^{-1} $ meets the definition of $ |B| \leq |A| $.

    Hence proved.
\end{solution}


\end{questions}
\end{sloppypar}
\end{document}

