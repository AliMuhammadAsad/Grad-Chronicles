\documentclass[a4paper]{exam}

\usepackage{amsmath,amssymb, amsthm}
\usepackage[a4paper]{geometry}
\usepackage{hyperref}
\usepackage{mdframed}

\title{Weekly Challenge 12: Closure of Decidable Languages}
\author{CS 212 Nature of Computation\\Habib University\\Ali Muhammad Asad - aa07190}
\date{Fall 2023}

% \theoremstyle{theorem}
\newtheorem{theorem}{Theorem}

% \theoremstyle{claim}
\newtheorem{claim}{Claim}

\qformat{{\large\bf \thequestion. \thequestiontitle}\hfill}
\boxedpoints

\printanswers

\begin{document}
\maketitle

\begin{questions}

  \titledquestion{Put Together}

  The \textit{put-together} operation, $f$, is defined on strings, $u=u_1u_2\ldots u_m$ and $v=v_1v_2\ldots v_n$, and extended to languages, $L_1$ and $L_2$, over an alphabet, $\Sigma$, as follows.
  \begin{align*}
    f(u,v)      & = u_1u_2\ldots u_mv_1v_2\ldots v_n      \\
    f(L_1, L_2) & = \{ f(u,v) \mid u\in L_1, v\in L_2 \}.
  \end{align*}

  Prove or disprove the following claim.
  \begin{claim}
    The class of decidable languages is closed under the put-together operation.
  \end{claim}

  \begin{solution}
    % A language is decidable if there exists a Turing Machine that decides (halts) on any stirng given to the machine; either accepting or rejecting.
    Let $L_1$ and $L_2$ be two decidable languages. Let $M_1$ and $M_2$ be the Turing Machines that decide $L_1$ and $L_2$ respectively. Let $L$ be the language $ f(L_1, L_2) $. Then we can construct a Turing Machine $M$ that decides $L$. For any given string $w$, the machine $M$ needs to determine if there exists strings $u$ and $v$ such that $f(u,v) = w$.

    Then $M$ works as follows: \vspace*{-2mm} \begin{itemize}
      \item As $w = w_1 w_2 w_3 ... w_p $ (where $p = m + n$) is a concatenation of $u$ and $v$, $M$ can try all possible ways of splitting $w$ into two strings $u$ and $v$ as follows; for each $i$ from 0 to $p$, consider the prefix of $w$ of length $i$ as a potential string $u$; $u = w_1 w_2 ... w_i$. The remaining part of the string, $ w_{i + 1} w_{i + 2} ... w_p $ is considered as a potential string $v$. [when $i = 0$, $ u = \emptyset $ and $v = w$, and when $ i = p $, $ u = w $ and $v = \emptyset$]. \vspace*{-2mm}
      \item Then for each potential string $u$ and $v$, simulate $M_1$ on $u$ and $M_2$ on $v$. \vspace*{-2mm}
      \item If $M_1$ accepts $u$ \textbf{and} $M_2$ accepts $v$, then $M$ accepts $w$. If either $M_1$ or $M_2$ rejects its respective string, then $M$ rejects $w$ for that particular string.
    \end{itemize} \vspace*{-2mm}
    Since both $M_1$ and $M_2$ are deciders, they will halt on all inputs. Therefore, $M$ will also halt on all inputs.

    If $w \in f(L_1, L_2)$, then there exists $ u \in L_1, v\in L_2 $ such that $ w = uv $. Machine $M$ will eventually simulate the correct split of $w$, and both $ M_1 $ and $M_2$ will accept, hence $M$ will accept. If $ w \notin f(L_1, L_2) $, then there are no such $ u $ and $ v $ that can both be accepted by $M_1$ and $M_2$ respectively. Thus, for all possible splittings of $w$, either $M_1$ or $M_2$ will reject. Therefore $M$ will also reject. Since $M$ accepts if and only if $w \in f(L_1, L_2)$, and rejects otherwise, $M$ decides $f(L_1, L_2)$. Therefore, $f(L_1, L_2)$ is decidable.

    Hence we can conclude that the class of decidable languages is closed under the put-together operation. \vspace*{-4mm}
    \begin{flushright}
      $\blacksquare$
    \end{flushright}
  \end{solution}

\end{questions}
\end{document}

%%% Local Variables:
%%% mode: latex
%%% TeX-master: t
%%% End:


% \textbf{Proof of Decidability:} \vspace*{-2mm} \begin{itemize}
%   \item If $w \in f(L_1, L_2)$, then by definition, there exists strings $ u \in L_1, v \in L_2 $ such that $ w = uv $. Machine $M$ will eventually simulate the correct split of $w$, and both $ M_1 $ and $M_2$ will accept, hence $M$ will accept. \vspace*{-2mm}
%   \item If $ w \notin f(L_1, L_2) $, then there are no such $ u $ and $ v $ that can both be accepted by $M_1$ and $M_2$ respectively. Thus, for all possible splittings of $w$, either $M_1$ or $M_2$ will reject. Therefore $M$ will also reject.
% \end{itemize} \vspace*{-2mm}
% Since $M$ accepts if and only if $w \in f(L_1, L_2)$, and rejects otherwise, $M$ decides $f(L_1, L_2)$. Therefore, $f(L_1, L_2)$ is decidable.