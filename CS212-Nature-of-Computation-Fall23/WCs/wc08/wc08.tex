\documentclass[a4paper]{exam}

\usepackage{amsmath,amssymb, amsthm}
\usepackage[a4paper]{geometry}
\usepackage{hyperref}
\usepackage{mdframed}

\title{Weekly Challenge 08: Context-Free Languages}
\author{CS 212 Nature of Computation\\Habib University\\Ali Muhammad Asad - aa07190}
\date{Fall 2023}

% \theoremstyle{theorem}
\newtheorem{theorem}{Theorem}

% \theoremstyle{claim}
\newtheorem{claim}{Claim}

\qformat{{\large\bf \thequestion. \thequestiontitle}\hfill}
\boxedpoints

\printanswers

\begin{document}
\maketitle

\begin{questions}

  \titledquestion{Closure}

  Given a language, $L$, the operation, $f$ is defined as $f(L) = \{w\mid vw \in L \text{ and } v,w\in\Sigma^*\}$.

  Prove or disprove the given claim.

  \begin{claim}
    The class of context-free languages is closed under $f$.
  \end{claim}

  \begin{solution}

    The operation $f$ generates a language consisting of all strings $w$ such that for some string $v$, the concatenation of $v$ and $w$ ($vw$) is in the original language. \\ In essence, it extracts a suffix $w$ from a string $vw$ that is in the original language $L$.

    % Given a context-free language $L$, let the PDA for $L$ be $P$.

    % Let the resulting language $f(L)$ be $L'$. Then we can construct a PDA $P'$ for $L'$ as follows: \vspace*{-2mm}
    % \begin{itemize}
    %   \item Create a copy of the PDA $P$ and name it as $P_c$ which would have the same transitions as $P$. $P$ and $P_c$ will be combined to form $P'$.
    %   \item Modify the input part of transitions in $P_c$ to $\varepsilon$ without changing the stack symbol. If the input transition has $ 0, 1 \rightarrow \varepsilon $ modify it to $ \varepsilon, 1 \rightarrow \varepsilon $. The input in the transition $ 0, 1 \rightarrow \varepsilon $ is 0 and it is changed $ \varepsilon, 1 \rightarrow \varepsilon $ where the stack symbol $ \varepsilon $ is unchanged. In this step, just change the input part of each transition irrespective of the stack symbol.
    %   \item For each state in $P_c$, add a new transition $ \varepsilon, \varepsilon \rightarrow \varepsilon $ to the corresponding state in $P$. This that for the input $ \varepsilon $ and stack symbol $ \varepsilon $, the top of the stack will be $ \varepsilon $. This step simply connects two PDAs.
    %   \item The start state of $P_c$ should be the start state of the whole PDA $P'$. Thus, $P'$ is the combination of the two PDAs $P$ and $P_c$.
    % \end{itemize}

    % The PDA $P'$ simply ignores the alphabet of $v$ and starts functioning when it identifies the first alphabet of $w$ from which the second part of the PDA $P'$ ($P$) accepts the substring $w$ - the suffix. Thus, all suffix $w$ of the string belong to the language $L$ will be accepted by $L'$.

    Given a context-free language $L$, let the PDA for $L$ be $P$.
    Let the resulting language $f(L)$ be $L'$. Then we can construct a PDA $P'$ for $L'$ as follows: \vspace*{-2mm}
    \begin{itemize}
      \item Create an identical copy of $P$ and name it as $P_c$, having the same transitions as $P$. $P$ and $P_c$ will be combined to form $P'$; $P_c$ recognizes the prefix $v$. \vspace*{-2mm}
      \item Modify the input part of transitions in $P_c$ to $\varepsilon$ without changing the stack symbol. For example, if the input transition has $ 0, 1 \rightarrow \varepsilon $ modify it to $ \varepsilon, 1 \rightarrow \varepsilon $. The input in the transition $ 0, 1 \rightarrow \varepsilon $ is 0 and it is changed $ \varepsilon, 1 \rightarrow \varepsilon $ where the stack symbol $ \varepsilon $ is unchanged. This step ignores the input symbols when transitioning in $P_c$ as we want $P_c$ to recgonize the prefix $v$, so the input symbols don't matter to it and it doesn't consume the input symbols. \vspace*{-2mm}
      \item For each state in $P_c$, add a new transition $ \varepsilon, \varepsilon \rightarrow \varepsilon $ to the corresponding state in $P$. This step connects each state in $P_c$ to its corresponding state in $P$, essentially connecting $P_c$ to $P$. It says that when $P_c$ is done recognizing the prefix $v$ and is ready to look at the suffix $w$, it transitions to the corresponding state in $P$. \vspace*{-2mm}
      \item The start state of $P_c$ should be the start state of the whole PDA $P'$. So when we start processing an input string, first we go through $P_c$ which recognizes the prefix $v$ and then we go through $P$ which recognizes the suffix $w$. Thus, $P'$ is the combination of the two PDAs $P$ and $P_c$.
    \end{itemize}
    Then, by this construction, $P_c$ `ignores' some prefix of the input string and then processes the rest of the string normally through $P$. Hence recognizing all strings $w$ after removing some prefix $v$ from the original string $vw$ in the language.

    Since a PDA recognizes $f(L)$, the class of context-free languages is closed under $f$. 
    \begin{flushright}
      $\blacksquare$
    \end{flushright}

  \end{solution}

\end{questions}
\end{document}

%%% Local Variables:
%%% mode: latex
%%% TeX-master: t
%%% End:
