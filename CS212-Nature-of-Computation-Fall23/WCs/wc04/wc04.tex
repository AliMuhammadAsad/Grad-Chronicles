\documentclass[a4paper]{exam}

\usepackage{amsmath,amssymb, amsthm}
\usepackage[a4paper]{geometry}
\usepackage{hyperref}
\usepackage{mdframed}
\usepackage{enumitem}

\title{Weekly Challenge 04: Regular Expressions}
\author{CS 212 Nature of Computation\\Habib University}
\date{Fall 2023}

\theoremstyle{definition}
\newtheorem{definition}{Definition}

% \theoremstyle{claim}
\newtheorem{claim}{Claim}

\qformat{{\large\bf \thequestion. \thequestiontitle}\hfill}
\boxedpoints


%%%%%%%% Uncomment to include your solution.
\printanswers

\begin{document}
\maketitle

\begin{questions}

\titledquestion{Closures}

Given a language, $L$, and the definitions below, prove or disprove the given claim.

\begin{definition}[Kleene closure]
  $L^*=\{ u_1u_2u_3\ldots u_n \mid \text{ each }u_i\in L, n\ge 0\}$
\end{definition}
\begin{definition}[Positive closure]
  $L^+=\{ u_1u_2u_3\ldots u_n \mid \text{ each } u_i\in L, n\ge 1\}$
\end{definition}

\begin{claim}
  $(L^+)^*=(L^*)^+$
\end{claim}  

\begin{solution}
% By the definition, for any arbitrary language $L$: \\ 
% \hspace*{3mm} - $L^*$ includes all possible concatenations of zero or more strings from $L$, which includes \hspace*{5mm} the empty string $\varepsilon$ \\ 
% \hspace*{3mm} - $L^+$ includes all possible concatenations of one or more strings from $L$
% From the definition of Kleene Closure, and the Positive Closure, one can deduce that $L^* = L^+ \cup \varepsilon$ since $L^*$ includes all possible concatenations of zero or more strings from $L$, which includes the empty string $\varepsilon$. 
% From the definition, it is obvious that $ L^* = L^+ \cup \{\varepsilon\} $
\textbf{Claim:} $(L^+)^*=(L^*)^+$ 
\vspace*{-2mm}
\begin{enumerate}[leftmargin=*]
  \item[1)] $ (L^+)^* \subseteq (L^*)^+ $ \\
  Consider any arbitrary string $s$ in $ (L^+)^* $. $s$ is composed of zero or more stirngs from $L^+$, which includes the empty string $\varepsilon$ by the definition of Kleene Closure. Then $s$ can be represented as a concatenation of other strings (substrings) \vspace*{-2mm}$$ s = s_1 s_2 s_3 ... s_n \text{ where each } s_i \in L^+, \text{ and } n \geq 0 \vspace*{-2mm}$$ 
  Now for any $s_i$, $ s_i $ must exist in $L^*$ because $L^*$ includes all strings (zero or more) from $ L^+ $. So each $s_i \text{ also exists in } L^*$, including the empty string as by the definition, $L^*$ also contains zero strings which amounts to the empty string $\varepsilon$. Then $ s = s_1 s_2 s_3 ... s_n \in (L^*)^+, \;\;\forall s \in (L^+)^*$ since the Positive Closure will be a concatenation of all strings from $L^*$ which will include $\varepsilon$ since $\varepsilon$ is a member of $L^*$. Hence, for any arbitrary string $s$ in $ (L^+)^* $, $s \in (L^*)^+$, which implies that $ (L^+)^* \subseteq (L^*)^+ $.

  % Now consider each $s_i$. Whether $ s_i = \varepsilon $ or $ s_i $ is not the empty string, $ s_i $ must exist in $L^*$ because $L^*$ includes all strings from $ L^+ $ and the empty string $\varepsilon$. So each $s_i \in L^*$ which includes $\varepsilon$ as a member of the set. Then $ s = s_1 s_2 s_3 ... s_n \in (L^*)^+$ since the Positive Closure will be a concatenation of all strings from $L^*$. Hence, for any arbitrary string $s$ in $ (L^+)^* $, $s \in (L^*)^+$, which implies that $ (L^+)^* \subseteq (L^*)^+ $.
  
  \item[2)] $ (L^*)^+ \subseteq (L^+)^* $ \\ 
  Consider any arbitrary string $s$ in $ (L^*)^+ $. $s$ is composed of one or more strings from $L^*$, which includes the empty string $\varepsilon$ due to the Kleene Closure. Then $s$ can be represented as a concatenation of other strings (substrings) \vspace*{-2mm}$$ s = s_1 s_2 s_3 ... s_n \text{ where each } s_i \in L^*, \text{ and } n \geq 1 \vspace*{-2mm}$$
  Now for any $s_i$ which is a component of $s$, if $s_i \neq \varepsilon$, then $s_i \in L^+$ and subsequently in $ (L^+)^* $. However, due to the application of Kleene Closure on $L^+$, the resulting language will have $\varepsilon$ as a member, hence $ s_i \in (L^+)^*$. Then each $s_i \in L^+ \text{ where } s_i \neq \varepsilon$, however, $ \varepsilon \in (L^+)^* $. Therefore, $ s = s_1 s_2 s_3 ... s_n \in (L^+)^*, \;\forall s \in (L^*)^+$ since its a concatenation of possible strings from $L^+$ including the empty string by the definition. Hence, for any arbitrary string $s$ in $ (L^*)^+ $, $s \in (L^+)^*$, which implies that $ (L^*)^+ \subseteq (L^+)^* $.
  
  % From the definition, $ (L^+) $ contains all possible concatenations of one or more strings from $L$. Consider any arbirary string $s$ from $ (L^+)^* $. Since $s$ is a concatenation of strings, then $s$ can be broken down into substrings \vspace*{-2mm}$$ s  = s_1s_2s_3...s_n \text{ where } s_i \in L^+ \text{ for } i = 1, 2, 3, ..., n $$

  % Consider any arbitray string $s$ from $(L^+)^*$. By the definition of the Kleene Closure, $s$ is composed of zero or more strings from $ L^+ $. \\ Then consider each substring of $s$ such that $ s = s_1s_2s_3...s_n $ where $ s_i \in  $. 
\end{enumerate}
Since $ (L^+)^* \subseteq (L^*)^+ \text{ and } (L^*)^+ \subseteq (L^+)^* $, then $ (L^+)^* = (L^*)^+ $.
\begin{flushright}
  $\blacksquare$
\end{flushright}

\end{solution}
  
\end{questions}
\end{document}

%%% Local Variables:
%%% mode: latex
%%% TeX-master: t
%%% End:
