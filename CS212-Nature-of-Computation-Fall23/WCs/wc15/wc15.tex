\documentclass[a4paper]{exam}

\usepackage{amsmath,amssymb, amsthm}
\usepackage[a4paper]{geometry}
\usepackage{hyperref}

% \theoremstyle{theorem}
\newtheorem{theorem}{Theorem}

% \theoremstyle{claim}
\newtheorem{claim}{Claim}

\printanswers

\qformat{{\large\bf \thequestion. \thequestiontitle}\hfill}
\boxedpoints

\title{Weekly Challenge 15: Complexity Classes}
\author{CS 212 Nature of Computation\\Habib University\\Ali Muhammad Asad - aa07190}
\date{Fall 2023}

\begin{document}
\maketitle

\begin{questions}

  \titledquestion{Checking Primality}

  Explain succinctly why the language recognized by the following Turing machine, $M$, does not belong to P. Assume the input to be a binary representation of a number.

  $M$ = On input $n$:
  \begin{enumerate}
    \item Check if $2$ divides $n$, if so \textit{reject}.
    \item Repeat Step 1 for all numbers less than $n$. That is, check if $3$ divides $n$. If so \textit{reject}, otherwise check if $4$ divides $n$, if so \textit{reject}, and so on.
    \item If all numbers less than $n$ have been checked, \textit{accept}.
  \end{enumerate}
  \begin{solution}
    \texttt{P} is the class of languages that are decidable in polynomial time on a single tape deterministic Turing Machine. $ \text{\texttt{P}} = \displaystyle\bigcup_k \text{TIME}(n^k) $

    Essentially, the language of $M$, $L(M)$, is the set of the binary representation of all prime numbers. It takes as input $n$ which is the binary representation of a number $x$, and rejects if $x$ is not prime, and accepts if $x$ is prime.

    The algorithm used by $M$ is quite a straightforward, brute-force approach. The time complexity of this approach, intuitively, is $ O(n) $ in terms of the number of divisions $M$ performs. However, considering the length of the input (the number of bits representing $n$), let's say $k$ bits, $n$ is at most $ 2^k $. So in terms of the input size that $M$ has to compute on, the compelexity is proportional to $ 2^k $, and thus is in $ O(2^k) $ which is exponential. Therefore, the language of $M$ is not polynomial, and hence $M$ does not belong to \texttt{P}.

    \begin{flushright}
      $\blacksquare$
    \end{flushright}

    \textit{*Note that the complexity of deciding whether a number is a Prime or not is in polynomial time. But this particular algorithm on this particular machine is not.}

    % The machine, $M$, takes an input $n$ that represents the binary representation of a number $x$. It checks if $n$ is divisible by any number from 2 to $n-1$. If $n$ is divisible by any of these numbers, $M$ rejects, else it accepts. Therefore, the language recognized by $M$, $L(M)$, is the set of the binary representation of all prime numbers.

    % Assuming that the input $n$ is the binary representation of a number, where our number is $k$, then the number of bits required to represent $k$ is $ \log_2 k $. 
    % Then as we increase $k$, the number of bits required to represent $k$ increases logarithmically. The number of operations required by $M$ is proportional to $ \log_2 k $. Then, at most, $M$ has to check all numbers less than equal to $k$ to see if they divide $k$. Then $M$ has complexity proportional to $k$, so it has time complexity $ O(k) $ which is exponential in the size of the input $n$ (since length of $n$ is at most $ \log_2 k $). Hence, $M$ does not belong to \texttt{P} since \texttt{P} is the set of problems that can be solved in polynomial time in the size of the input.

    % Hence, $M$ does not belong to \texttt{P}.
    % \vspace*{-4mm}\begin{flushright}
    %   $\blacksquare$
    % \end{flushright}
  \end{solution}
\end{questions}
\end{document}

%%% Local Variables:
%%% mode: latex
%%% TeX-master: t
%%% End:
