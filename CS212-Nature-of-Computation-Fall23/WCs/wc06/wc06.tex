\documentclass[a4paper]{exam}

\usepackage{amsmath,amssymb, amsthm}
\usepackage[a4paper]{geometry}
\usepackage{hyperref}
\usepackage{mdframed}
\usepackage{mathtools}

\title{Weekly Challenge 06: Context-Free Languages}
\author{CS 212 Nature of Computation\\Habib University\\Ali Muhammad Asad (aa07190)}
\date{Fall 2023}

% \theoremstyle{theorem}
\newtheorem{theorem}{Theorem}

% \theoremstyle{claim}
\newtheorem{claim}{Claim}

\qformat{{\large\bf \thequestion. \thequestiontitle}\hfill}
\boxedpoints

% \usepackage{draftwatermark}
% \SetWatermarkText{Sample Solution}
% \SetWatermarkScale{3}
\printanswers

\begin{document}
\maketitle

\begin{questions}

\titledquestion{Closure}

Given the following theorem, prove or disprove the given claim.

\begin{theorem}
The class of context-free languages is not closed under intersection.
\end{theorem}
\begin{claim}
The class of context-free languages is not closed under complementation.
\end{claim}

\begin{solution}
	Let $L_1$ and $L_2$ be any arbitrary context-free languages (CFLs). By the theorem given above, we know that the intersection for the class of CFLs is not closed under intersection.

	Then $ L_1 \cap L_2 $ is not closed. By De Morgan's Law, $ L_1 \cap L_2 = \overline{\overline{L_1} \cup \overline{L_2}} $. If CFLs were closed under both union and complement, then they would be closed under intersection. 
	% Since they are not closed under intersection, it suffices to show that CFLs are closed under union, which would imply that they are not closed under complementation, therefore, not closed under intersection.

	\textbf{Closure under Union} \\ 
	Let $C_1$ and $C_2$ be two CFLs recognized by $ G_1 = (V_1, \sum, R_1, S_1) $ and $ G_2 = (V_2, \sum, R_2, S_2) $ respectively. Assume that $ V_1 \cap V_2 = \emptyset $; if this assumption is not true, rename the variables of one of the grammars to make this condition true. 
	
	We need to construct a grammar $ G = (V, \sum, R, S) $ such that $ L(G) = L(G_1) \cup L(G_2) $. We can construct $ G $ as follows: \vspace*{-3mm} \begin{itemize}
		\item $ V = V_1 \cup V_2 \cup \{ S \}, $ where $ S \notin V_1 \cup V_2 $ (and $ V_1 \cap V_2 = \emptyset $) \vspace*{-2mm}
		\item $ R = R_1 \cup R_2 \cup \{ S \rightarrow S_1 \mid S_2 \} $ 
	\end{itemize}

	% Proof 2
	The above grammar combines the two grammars $ G_1 $ and $ G_2 $ into a single grammar $ G $, by adding a new start variable $ S $ and a new production rule $ S \rightarrow S_1 \mid S_2 $, which allows us to derive strings in $ L(G_1) \cup L(G_2) $.

	Consider an arbitrary string $w \in L(G)$. Then either $ w \in L(G_1) $, or $ w \in L(G_2) $ which implies that either $ S_1 \overset{*}{\implies}_{\hspace*{-1mm}G} w $ or $ S_2 \overset{*}{\implies}_{\hspace*{-1mm}G} w$. Since $G$ has the production $ S \rightarrow S_1 \mid S_2 $, we can derive $ w $ from $ S $ by using this rule. Hence $ S \overset{*}{\implies}_{\hspace*{-1mm}G} w $, which implies that $ w \in L(G) $. Hence, if $ w \in L(G) $, then $ w \in L(G_1) \cup L(G_2) $.

	Conversely, we have $ S \overset{*}{\implies}_{\hspace*{-1mm}G} w $. Then either $ S_1 \overset{*}{\implies}_{\hspace*{-1mm}G} w $ or $ S_2 \overset{*}{\implies}_{\hspace*{-1mm}G} w $. Since $ V_1 \cap V_2 = \emptyset, w $ is either derived from $S_1$ using $R_1$, or from $S_2$ using $R_2$. Therefore, $ w \in L_1 \cup L_2 $. Hence if $ w \in L(G_1) \cup L(G_2) $, then $ w \in L(G) $.

	% Proof 1
	% Consider an arbitrary string $w \in L(G)$. Then there is a derivation $ S \overset{*}{\implies}_{\hspace*{-1mm}G} w $. Since the only rules involving $S$ are $ S \rightarrow S_1 $ and $ S \rightarrow S_2 $, this derivation is either of the form $ S \implies_{\hspace*{-2mm}G} \; S_1 \overset{*}{\implies}_G w $ or $ S \implies_{\hspace*{-2mm}G} \; S_2 \overset{*}{\implies}_{\hspace*{-1mm}G} w $. Since the only rules for variables in $V_1$ are those belonging to $R_1$, and since $ S_1 \overset{*}{\implies}_{\hspace*{-1mm}G} w $, we have $ S_1 \overset{*}{\implies}_{\hspace*{-1mm}G_1} w $ and so $ w \in L(G_1) $. If the derivation $ S \overset{*}{\implies}_{\hspace*{-1mm}G} $ is of the form $ S \implies_{\hspace*{-2mm}G} \; S_2 \overset{*}{\implies}_{\hspace*{-1mm}G} w $, then the same reasoning follows and we can conclude that $ w \in L(G_2) $. Hence if $ w \in L(G) $, then $ w \in L(G_1) \cup L(G_2) $. 

	% Conversely, consider an arbitrary string $w \in L(G_1) \cup L(G_2)$. Then either $ w \in L(G_1) $ or $ w \in L(G_2) $. If $ w \in L(G_1) $, then there is a derivation $ S_1 \overset{*}{\implies}_{\hspace*{-1mm}G_1} w $, and so $ S \overset{*}{\implies}_{\hspace*{-1mm}G} w $, where $ S \implies_{\hspace*{-2mm}G} \; S_1 $. Similarly, if $ w \in L(C_2) $, then there is a derivation $ S_2 \overset{*}{\implies}_{\hspace*{-1mm}G_2} w $, and so $ S \overset{*}{\implies}_{\hspace*{-1mm}G} w $, where $ S \implies_{\hspace*{-2mm}G} \; S_2 $. Hence if $ w \in L(G_1) \cup L(G_2) $, then $ w \in L(G) $.

	Hence proved that CFLs are closed under Union.

	\textbf{Closure under Complementation} \\
	We already estbalished that $ L_1 \cap L_2 = \overline{\overline{L_1} \cup \overline{L_2}}$, and that the intersection is not closed.
	
	Assume that CFLs were closed under complementation. Then $ \overline{L_1} $ and $ \overline{L_2} $ are closed. And since CFLs are closed under union, then $ \overline{L_1} \cup \overline{L_2} $ is a CFL, which implies that $ \overline{\overline{L_1} \cup \overline{L_2}} $ is a CFL. 
	
	However, $ \overline{\overline{L_1} \cup \overline{L_2}} = L_1 \cap L_2 $, which means that CFLs are closed under intersection, hence we arrive at a contradiction, as by the theorem given above, CFLs are not closed under intersection. 

	Since CFLs are closed under union, then they must not be closed under complementation. Hence proved.
	\begin{flushright}
		$ \blacksquare $
	\end{flushright}
\end{solution}

\end{questions}
\end{document}

%%% Local Variables:
%%% mode: latex
%%% TeX-master: t
%%% End:
