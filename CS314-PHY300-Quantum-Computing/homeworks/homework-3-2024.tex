\documentclass{article}

\usepackage{url}
\usepackage{fancyhdr}
\usepackage{extramarks}
\usepackage{amsmath}
\usepackage{amsthm}
\usepackage{amsfonts}
\usepackage{tikz}
\usetikzlibrary{3d}
\usepackage[plain]{algorithm}
\usepackage{algpseudocode}
\usepackage{braket}
\usepackage{enumerate}
\usepackage{paralist}

%
% Basic Document Settings
%

\topmargin=-0.45in
\evensidemargin=0in
\oddsidemargin=0in
\textwidth=6.5in
\textheight=9.0in
\headsep=0.25in

\linespread{1.1}

\pagestyle{fancy}
\lhead{Habib University}
\chead{\hmwkClass, \hmwkTitle}
\rhead{\firstxmark}
\lfoot{\lastxmark}
\cfoot{\thepage}

\renewcommand\headrulewidth{0.4pt}
\renewcommand\footrulewidth{0.4pt}

\setlength\parindent{0pt}

%
% Create Problem Sections
%

\newcommand{\enterProblemHeader}[1]{
	\nobreak\extramarks{}{Problem \arabic{#1} continued on next page\ldots}\nobreak{}
	\nobreak\extramarks{Problem \arabic{#1} (continued)}{Problem \arabic{#1} continued on next page\ldots}\nobreak{}
}

\newcommand{\exitProblemHeader}[1]{
	\nobreak\extramarks{Problem \arabic{#1} (continued)}{Problem \arabic{#1} continued on next page\ldots}\nobreak{}
	\stepcounter{#1}
	\nobreak\extramarks{Problem \arabic{#1}}{}\nobreak{}
}

\setcounter{secnumdepth}{0}
\newcounter{partCounter}
\newcounter{homeworkProblemCounter}
\setcounter{homeworkProblemCounter}{1}
\nobreak\extramarks{Problem \arabic{homeworkProblemCounter}}{}\nobreak{}

%
% Homework Problem Environment
%
% This environment takes an optional argument. When given, it will adjust the
% problem counter. This is useful for when the problems given for your
% assignment aren't sequential. See the last 3 problems of this template for an
% example.
%
\newenvironment{homeworkProblem}[1][-1]{
	\ifnum#1>0
	\setcounter{homeworkProblemCounter}{#1}
	\fi
	\section{Problem \arabic{homeworkProblemCounter}}
	\setcounter{partCounter}{1}
	\enterProblemHeader{homeworkProblemCounter}
}{
	\exitProblemHeader{homeworkProblemCounter}
}

%
% Homework Details
%   - Title
%   - Due date
%   - Class
%   - Section/Time
%   - Instructor
%   - Author
%

\newcommand{\hmwkTitle}{Homework\ \#3}
\newcommand{\hmwkDueDate}{December 07, 2024, 11.59pm}
\newcommand{\hmwkClass}{CS 314/PHYS 300: Quantum Computing}
\newcommand{\hmwkClassInstructor}{Dr. Faisal Alvi}
\newcommand{\hmwkAuthorName}{\textbf{Student 1 Name, ID} \and \textbf{Student 2 Name, ID}}

%
% Title Page
%

\title{
	\vspace{2in}
	\textmd{\textbf{\hmwkClass:\\ \hmwkTitle}}\\
	\normalsize\vspace{0.1in}\small{\hmwkClassInstructor}\\
	\normalsize\vspace{0.1in}\small{Due\ on\ \hmwkDueDate}\\
	\vspace{3in}
}

\author{\hmwkAuthorName}
\date{}

\renewcommand{\part}[1]{\textbf{\large Part \Alph{partCounter}}\stepcounter{partCounter}\\}

%
% Various Helper Commands
%

% Useful for algorithms
\newcommand{\alg}[1]{\textsc{\bfseries \footnotesize #1}}

% For derivatives
\newcommand{\deriv}[1]{\frac{\mathrm{d}}{\mathrm{d}x} (#1)}

% For partial derivatives
\newcommand{\pderiv}[2]{\frac{\partial}{\partial #1} (#2)}

% Integral dx
\newcommand{\dx}{\mathrm{d}x}

% Alias for the Solution section header
\newcommand{\solution}{\textbf{\large Solution}}

% Probability commands: Expectation, Variance, Covariance, Bias
\newcommand{\E}{\mathrm{E}}
\newcommand{\Var}{\mathrm{Var}}
\newcommand{\Cov}{\mathrm{Cov}}
\newcommand{\Bias}{\mathrm{Bias}}

\begin{document}
	
\maketitle
	
\pagebreak
	
\begin{homeworkProblem}
(15 points) [\textbf{Grover's Algorithm - States}] In the version of Grover's Algorithm that we studied in the class (where our initial state was $\ket{\psi}$ as a uniform superposition of all states) we used:
\begin{itemize}
	\item an oracle `$O$' to invert the sign of the required state.
	\item a Grover Diffusion Operator `$G$' denoted by $2\ket{\psi}\bra{\psi} - I$ to amplify the amplitude of the inverted state. 
\end{itemize}

\begin{enumerate}[(a)]
\item (10 points) Show that the Grover's Diffusion Operator `$G$'  
\begin{center}
$2\ket{\psi}\bra{\psi} - I$ 
\end{center}
is equivalent to 
\begin{center}
$H^{\otimes n} (2\ket{0}\bra{0} - I) H^{\otimes n}$.
\end{center}
\item  (5 points) Construct (draw) a partial circuit for the Grover's Algorithm that uses this construction (i.e. $H^{\otimes n} (2\ket{0}\bra{0} - I) H^{\otimes n}$) used for the Grover Diffusion Operator.	
\end{enumerate}
\end{homeworkProblem}
	
\begin{homeworkProblem} (10 points) [\textbf{Grover's Algorithm - Sample Run - Adapted from Cambridge University Course on Quantum Computation] \textit{https://www.cl.cam.ac.uk/teaching/exams/pastpapers/y2023p8q11.pdf}}] 

\vspace{5pt}	

Let there be a database containing 32 elements, indexed by the binary numbers 00000 to 11111. A single element 00110 is marked.
\begin{enumerate}[(a)]
\item (3 points) If Grover’s search algorithm is applied to find the marked element, what should the initial state be set to, and what is the state after a single Grover iterate has been applied? 
\item (4 points) To find the marked element with maximum probability requires \textit{N} iterates in total. What is the value of \textit{N}, and what is the probability of correctly finding the marked element? Show the complete working at how did you arrive at the value of probability.
\item (3 points) If the algorithm is instead run with 3\textit{N} iterates in total, what is the probability of correctly finding the marked element? Comment on your answer.
\end{enumerate}
	
\end{homeworkProblem}

\begin{homeworkProblem} (10 points) \textbf{[Eigenvalues and Eigenvectors]}  Find the Eigenvalues and Eigenvectors of:
\begin{enumerate} [(a)]
		\item (4 points) The `$X$' (or the not) quantum operator given by
		$\begin{bmatrix}
			0 & 1 \\
			1 & 0 
		\end{bmatrix}.$
		\item (6 points) The `$H$' (or the Hadamard) Operator given by: $\begin{bmatrix}
			1/\sqrt{2} & 1/\sqrt{2} \\
			1/\sqrt{2} & -1/\sqrt{2} 
		\end{bmatrix}.$
\end{enumerate}
The Eigenvalues of a square matrix $A$ can be calculated by solving for the eigenvalues `$\lambda$' in the equation 	`$det(A - \lambda I) = 0$', and then solving for the eigenvector $v$ using the equation `$(\lambda I - A)v = 0$'. Since we are working with qubits, ensure that the norm of your eigenvectors equals to 1. Show all your work.
\end{homeworkProblem}

\begin{homeworkProblem} (15 points) \textbf{[Inverse Quantum Fourier Transform Using Pauli Matrices and $H$]}
\begin{enumerate}[(a)]
\item (10 points) Using the circuit for \textit{inverse} quantum fourier transform, construct (draw) the circuit for the \textit{inverse} fourier transform for 2 bits using the Hadamard, controlled-\textit{T} and controlled-NOT gates only. Show that the original circuit for the inverse QFT and your circuit are equivalent.

\item (5 points) Write a Qiskit function to implement this inverse quantum fourier transform. Run your circuit for all combinations of two qubits and show that the output is correct. 
\end{enumerate}
\end{homeworkProblem}
\textbf{Submission Guidelines:}
\begin{enumerate}
	\item Submit your solutions involving content, proofs, etc. as a latex pdf. 
	\item Submit the programming part as .ipynb files or as links to Google Colab pages. 
	\item Submit the entire HW as a zipped file.
\end{enumerate}

\end{document}

