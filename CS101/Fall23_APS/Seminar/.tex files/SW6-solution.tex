\documentclass[a4paper]{exam}

\usepackage{amssymb}
\usepackage{draftwatermark}
\usepackage[a4paper]{geometry}
\usepackage{mdframed}
\usepackage{tabularx}
\usepackage[usenames,dvipsnames,table]{xcolor}
\usepackage{fancyvrb}
\usepackage{pythonhighlight}
\SetWatermarkText{SAMPLE SOLUTION}
\SetWatermarkScale{2}

\setlength\parindent{0pt}
\newcommand{\class}{CS 101}
\newcommand{\term}{Fall 2023}
\newcommand\heading[1]{\textbf{#1}}
\newcommand\alert[2]{\centerline{\textbf\large{\underline{\color{#1}{#2}}}}}
\newcommand\inz{\in \mathbb{Z}}
\newcommand\inn{\in \mathbb{N}}

\runningheader{\class, \term}{Worksheet: Functions }{HU ID: \rule{75pt}{0.5pt}}
\runningheadrule
\runningfootrule
\runningfooter{}{Page \thepage\ of \numpages}{}

\qformat{{\large\bf \thequestion. \thequestiontitle}\hfill}
\boxedpoints

\title{Worksheet: Functions}
\author{\class\ Algorithmic Problem Solving}
\date{\term}

% -------------------
% Content
% -------------------
\begin{document}
\maketitle

Name(s): \hrulefill\\[5pt]
HU ID \textit{\small(e.g., xy01042)}: \hrulefill\\

\begin{questions}

    % Question 1
    \titledquestion{AC in summer!}

    Three of your friends are sitting in a room - Ali, Raahim, Saad. They need to decide on the temperature to set on their AC. Everyone has a demand.
    \begin{itemize}
        \item Ali wants the temperature to be at least A degrees.
        \item Raahim wants the temperature to be at most B degrees.
        \item Saad wants the temperature to be atleast C degrees.
    \end{itemize}

    Write a function temperature which returns a "Yes" if they can decide on some temperature which fits all their demands or "NO", if no temperature fits all their demands.

    Make sure to call your function and to present the output use:

    print(functionname(parameters))

    \heading{Constraints}
    \begin{itemize}
        \item $ 1 \leq A,B,C \leq 200 $

    \end{itemize}

    \heading{Interaction}

    The input comprises a single line containing 3 space-separated integers denoting the values of $A, B, C$

    The output must contain either a "YES" or "NO".

    \heading{Sample}

    \rowcolors{2}{gray!25}{white}
    \begin{tabularx}{\textwidth}{|X|X|}
        \rowcolor{gray!50}
        \hline
        Input    & Output \\ \hline\hline
        30 35 25 & YES    \\\hline
        30 35 40 & NO     \\\hline
    \end{tabularx}

    In the first case, $(A,B,C)=(30,35,25)$. Ali wants the temperature to be $ \geq 30$, Raahim wants it to be $ \leq 35$ and Saad wants it to be $ \geq 25$. Temperatures from 30 - 35 satisfy their demands so the answer is a "YES".

    In the second case, $(A,B,C)=(30,35,40)$. Ali awants the temperature to be $ \geq 30$, Raahim wants it to be $ \leq 35$ and Saad wants it to be $ \geq 40$. There is no temperature that satisfies this.

    \heading{Exercise}

    In the space provided, indicate the outputs for the given inputs.

    \rowcolors{2}{gray!25}{white}
    \begin{tabularx}{\textwidth}{|X|X|}
        \rowcolor{gray!50}
        \hline
        Input    & Output \\ \hline\hline
        30 35 35 &    YES    \\\hline
        30 25 35 &    NO    \\\hline
        19 28 10 &    YES    \\\hline
    \end{tabularx}


    \heading{Problem Identification}\\
    Briefly explain the underlying problem you identified in the above question that led you to your solution.


   Input: A, B, C \\
   Output: ''Yes” if A is less than or equal to B, and B is greater than or equal to C,  ”NO” otherwise.

    \heading{Pseudocode}
    \begin{verbatim}
        def temperature(A,B,C):
            if A<=B and B>=C:
                return "YES"
            else:
                return "NO"
            
        print(temperature(A,B,C))
    \end{verbatim}

 

    \heading{Dry Run}\\
    Using any of the inputs provided in the Exercise section above, dry run your psuedocode in the space below.

With A=30, B=35, and C=35, the code runs as follows:

\begin{enumerate}
   \item The temperature function is called with A=30, B=35, and C=35.
    \item Inside the function, it checks if A (30) is less than or equal to B (35), which is true. It also checks if B (35) is greater than or equal to C (35), which is true as well.
    \item Since both conditions are true, the function returns "YES", which is also our expected output.
    
\alert{Green} {This means the applied logic is correct}
\end{enumerate}
  



    \titledquestion{Equalizing something}
    You have two integers A and B. In one operation you can chose any integer d, and make one of the following moves:
    \begin{itemize}
        \item add d to A and subtract d from B.
        \item Add d to B and subtract d from A.
    \end{itemize}

    You can make as many operations as you want.

    Write a function equalizer that returns yes if you can make A and B equal and no if you cannot.


    Make sure to call your function and to present the output use:

    print(functionname(parameters))

    \heading{Constraints}
    \begin{itemize}
        \item $ 1 \leq A,B \le 1000 $

    \end{itemize}

    \heading{Interaction}

    The input comprises a single line containing 2 space-separated integers denoting the values of $A$ and $B$.

    The output must contain either a Yes or a No.

    \heading{Sample}

    \rowcolors{2}{gray!25}{white}
    \begin{tabularx}{\textwidth}{|X|X|}
        \rowcolor{gray!50}
        \hline
        Input & Output \\ \hline\hline
        3 3   & Yes    \\\hline
        1 2   & No     \\\hline
    \end{tabularx}

    In the first case, $(A,B)=(3,3)$. A and B are already equal so no operations are needed hence a yes.

    In the second case, $(A,B)=(1,2)$. You cannot make any operations to make A and B equal.

   
    \heading{Exercise}

    In the space provided, indicate the outputs for the given inputs.

    \rowcolors{2}{gray!25}{white}
    \begin{tabularx}{\textwidth}{|X|X|}
        \rowcolor{gray!50}
        \hline
        Input & Output \\ \hline\hline
        1 5   & Yes    \\\hline
        90 23 & No     \\\hline
        31 22 & No     \\\hline
    \end{tabularx}

    \heading{Problem Identification}\\
    Briefly explain the underlying problem you identified in the above question that led you to your solution.

    Input: A, B \\
    Output: "Yes" if the absolute difference between A and B is an even number, "No" otherwise.


    \heading{Pseudocode}
    \begin{verbatim}
        def equalizer(A,B):
            d = A-B
            if d < 0:
                d = -d 
            if (d%2 == 0):
                return "Yes"
            else:
                return "No"
        
        print(equalizer(A,B))

    \end{verbatim}

    
    \heading{Dry Run}\\
    Using any of the inputs provided in the Exercise section above, dry run your pseudo code in the space below.


With $A=1$ and $B=5$, the code runs as follows:

The \texttt{equalizer} function is called with $A=1$ and $B=5$.

\begin{enumerate}
    \item Inside the function, it calculates the absolute difference between $A$ and $B$: $D = A - B$, which results in $D = 1 - 5$, making $D = -4$.
    \item The code then checks if $D$ is less than 0, which is true since $D$ is -4. It proceeds to make $d$ positive by taking its absolute value: $D = -D$, so $D$ becomes 4. 
    \item Next, it checks if $D$ is even, and in this case, 4 is indeed an even number.
    \item Since the condition $(D \% 2 == 0)$ is true, the function returns "Yes", which is also our expected output.
    
\end{enumerate}


\alert{Green} {This means the applied logic is correct}



    \titledquestion{Car Mileage matters doesn't it?}
    You want to go on a road trip and decide to rent a car to travel to a restaurant that is $N$ kilometers away. You can either rent a petrol car or a diesel car.

    One liter of petrol would cost you $X$ rupees and one liter of diesel costs $Y$ rupees. You can travel $A$ kilometers with one liter of petrol and $B$ kilometers with one liter of diesel.

    You can also buy petrol and diesel in any amount, not necessarily an integer.

    Write a function mileage which takes the parameters $N,X,Y,A$ and $B$ and returns diesel if diesel car is better petrol if petrol car is better and any of the costs of both cars are the same.

    Make sure to call your function and to present the output use:

    print(functionname(parameters))

    \heading{Constraints}
    \begin{itemize}
        \item $ 1 \leq N,X,Y,A,B \le 100 $

    \end{itemize}

    \heading{Interaction}

    The input comprises a single line containing 5 space-separated integers denoting the values of $N,X,Y,A,B$

    The output must contain either a petrol, diesel or any depending on criteria specified above.

    \heading{Sample}

    \rowcolors{2}{gray!25}{white}
    \begin{tabularx}{\textwidth}{|X|X|}
        \rowcolor{gray!50}
        \hline
        Input        & Output \\ \hline\hline
        20 10 8 2 4  & Diesel \\\hline
        50 12 12 4 3 & Petrol \\\hline
    \end{tabularx}

    In the first case, $(N,X,Y,A,B)=(20,10,8,2,4)$. The cost of traveling by the petrol car will be 100 rupees while that of using the diesel car will be
    40 rupees. Hence, diesel car is better.

    In the second case, $(N,X,Y,A,B)=(50,12,12,4,3)$.  The cost of traveling by the petrol car will be 150 rupees while that of using the diesel car will be 200 rupees. Hence, a petrol car is better.

     \heading{Exercise}

    In the space provided, indicate the outputs for the given inputs.

    \rowcolors{2}{gray!25}{white}
    \begin{tabularx}{\textwidth}{|X|X|}
        \rowcolor{gray!50}
        \hline
        Input        & Output \\ \hline\hline
        40 3 15 8 40 & Any    \\\hline
        21 9 9 2 9   & Diesel \\\hline
        20 12 20 8 2 & Petrol \\\hline
    \end{tabularx}

    \heading{Problem Identification}\\
    Briefly explain the underlying problem you identified in the above question that led you to your solution.
    
 Input: N, X, Y, A, B \\
 Output: 'Diesel' if the relative cost of petrol is higher, 'Petrol' if the relative cost of diesel is higher, 'Any' if both are equal.

    \heading{Pseudocode}
    
    \begin{verbatim}
        def mileage(N,X,Y,A,B):
            reqd_petrol = N/A 
            reqd_diesel = N/B
            petrol_cost = reqd_petrol * X 
            diesel_cost = reqd_diesel * Y 
            if petrol_cost < diesel cost:
                return "Petrol"
            elif petrol_cost == diesel_cost:
                return "Any
            else:
                return "Diesel"

        print(mileage(N,X,Y,A,B))
    \end{verbatim}

    \heading{Dry Run}\\
    Using any of the inputs provided in the Exercise section above, dry run your psuedocode in the space below.


With $N=40$, $X=3$, $Y=15$, $A=8$, and $B=40$, the code runs as follows:

\begin{enumerate}
    \item The \texttt{mileage} function is called with $N=40$, $X=3$, $Y=15$, $A=8$, and $B=40$.
    \item Inside the function, it calculates the required petrol in liters: $reqd\_petrol = \frac{N}{A} = \frac{40}{8} = 5$ liters.
    \item  It also calculates the required diesel in liters: $reqd\_diesel = \frac{N}{B} = \frac{40}{40} = 1$ liter.
    \item The cost of petrol is computed as: $petrol\_cost = reqd\_petrol \cdot X = 5 \cdot 3 = 15$ units.
    \item The cost of diesel is calculated as: $diesel\_cost = reqd\_diesel \cdot Y = 1 \cdot 15 = 15$ units.
    \item Next, the code compares the petrol cost and diesel cost. Since they are equal (petrol\_cost equals diesel\_cost), it returns "Any", which is also our expected output for these values.    
\end{enumerate}


\alert{Green} {This means the applied logic is correct}


    \titledquestion{Five is a great number.}
    You are given a positive integer $N$, you want to determine if its possible to rearrange the digits of $N$ and obtain a multiple of $5$.

    Write a function five that takes in the parameter $N$ and returns a yes if it is possible to rearrange the digits to obtain a multiple of 5 and no if it is not possible.

    Make sure to call your function and to present the output use:

    print(functionname(parameters))

    \heading{Constraints}
    \begin{itemize}
        \item $ 1 \leq N \le 100000$

    \end{itemize}

    \heading{Interaction}

    The input comprises a single integer which is the value of $N$

    The output must contain a either a Yes or No according to the above criteria

    \heading{Sample}

    \rowcolors{2}{gray!25}{white}
    \begin{tabularx}{\textwidth}{|X|X|}
        \rowcolor{gray!50}
        \hline
        Input & Output \\ \hline\hline
        115   & Yes    \\\hline
        119   & No     \\\hline
    \end{tabularx}

    In the first case, $N=115$, The number $115$ is divisible by 5 and needs no rearranging.

    In the second case, $N=119$, The number cannot be rearranged in any way to make it divisible by 5.

   \heading{Exercise}

    In the space provided, indicate the outputs for the given inputs.

    \rowcolors{2}{gray!25}{white}
    \begin{tabularx}{\textwidth}{|X|X|}
        \rowcolor{gray!50}
        \hline
        Input & Output \\ \hline\hline
        103   & Yes    \\\hline
        158   & Yes    \\\hline
        291   & No     \\\hline
    \end{tabularx}

    \heading{Problem Identification}\\
    Briefly explain the underlying problem you identified in the above question that led you to your solution.

Input: N \\
Output: "Yes" if N contains '0' or '5', "No" otherwise.


    \heading{Pseudocode}
    \begin{verbatim}
        def five(N):
            X = str(N)
            flag = 0
            for i in X:
                if (i=="0" or i=="5"):
                    flag = 1
            if flag == 0:
                return "No"
            else:
                return "Yes"

        print(five(N)
    \end{verbatim}

    \heading{Dry Run}\\
    Using any of the inputs provided in the Exercise section above, dry run your psuedocode in the space below.


With $N=103$, the code runs as follows:

\begin{enumerate}
    \item The \texttt{five} function is called with $N=103$.
    \item Inside the function, it converts the integer $N$ into a string $X$. In this case, $X$ becomes "103."
    \item The variable $flag$ is initialized to 0.
    \item The code then enters a loop that iterates through each character $i$ in the string $X.$
    \item During each iteration, it checks if $i$ is equal to "0" or "5." If it is, it sets $flag$ to 1.
    \item After iterating through all characters in $X$, if $flag$ remains 0, it returns "No." Otherwise, it returns "Yes."
    \item  In this specific case, when the loop checks the characters "1," "0," and "3," it finds that the character "0" is indeed equal to "0." so $flag$ is set to 1. As a result, the function returns "Yes", which is also our expected output 
\end{enumerate}


\alert{Green} {This means the applied logic is correct}

    \titledquestion{Happy String}
    You have a string $S$ with you, a string is happy if it contains a contiguous substring of length strictly greater than 2 in which all its characters are vowels.

    Write a function happy that takes in the parameter $S$ and returns if the string is happy or not.

    Note that, in english alphabetEnglishs are a,e, i,o, and u.

    Note: string will only contain lowercase letters.

    Make sure to call your function and to present the output use:


    print(functionname(parameters))

    \heading{Constraints}
    \begin{itemize}
        \item $ 3 \leq |S| \le 1000$

    \end{itemize}

    \heading{Interaction}

    The input comprises a string of variable length.

    The output must contain either Happy or Sad according to the criteria given above.

    \heading{Sample}

    \rowcolors{2}{gray!25}{white}
    \begin{tabularx}{\textwidth}{|X|X|}
        \rowcolor{gray!50}
        \hline
        Input     & Output \\ \hline\hline
        abxy      & Sad    \\\hline
        abcdeeafg & Happy  \\\hline
    \end{tabularx}

    In the first case, $S=abxy$, There is only one vowel a therefore the string is sad.

    In the second case, $S=abcdeeafg$, There is a 3 vowel substring eea therefore the string is happy.


  \heading{Exercise}

    In the space provided, indicate the outputs for the given inputs.

    \rowcolors{2}{gray!25}{white}
    \begin{tabularx}{\textwidth}{|X|X|}
        \rowcolor{gray!50}
        \hline
        Input   & Output \\ \hline\hline
        aeiou   & Happy  \\\hline
        abedfeg & Sad    \\\hline
        ewkiaou & Happy  \\\hline
    \end{tabularx}



    \heading{Problem Identification}\\
    Briefly explain the underlying problem you identified in the above question that led you to your solution.

Input: S \\
Output: "Happy" if three contiguous vowels are encountered in S, "No" otherwise.

    \heading{Pseudocode}
    \begin{verbatim}
        def happy(S):
            count = 0
            flag = False
            for i in S:
                if i in "aeiou":
                    count = count + 1
                    if count == 3:
                        flag = True 
                        break
                else:
                    count = 0
                    flag = False

            if flag == True:
                return "Happy"
            else:  
                return "Sad"

        print(happy(S))
    \end{verbatim}

    

 

    \heading{Dry Run}\\
    Using any of the inputs provided in the Exercise section above, dry run your psuedocode in the space below.

With $S = \textit{"abedfeg"}$, the code runs as follows:

\begin{enumerate}
    \item The `happy` function is called with $S = \textit{"abedfeg"}$.
    
    \item Inside the function, two variables are initialized:
    
    \begin{itemize}
        \item \texttt{count} is set to 0, which will be used to count consecutive vowels.
        
        \item \texttt{flag} is initialized as False, which will be used to indicate if three consecutive vowels are found.
    \end{itemize}
    
    \item The function enters a for loop that iterates through each character \texttt{i} in the string \texttt{S}.
    
    \item During each iteration, it checks if \texttt{i} is in the set of vowels "aeiou." 
    \item In this specific case with $S = \textit{"abedfeg"}$:
    
    \begin{itemize}
        \item The loop encounters "a" and increments \texttt{count} to 1.
        
        \item It then encounters "b," which is not a vowel. So, \texttt{count} is reset to 0, and \texttt{flag} is set to False.
        
        \item It continues with "e" and increments \texttt{count} to 1.
        
        \item When it encounters "d," which is not a vowel, \texttt{count} is reset to 0, and \texttt{flag} is set to False.
        
        \item The loop proceeds with "f" and "g," which are not vowels. So, \texttt{count} remains 0, and \texttt{flag} is still False.
        
        \item After iterating through all characters in \texttt{S}, the code checks \texttt{flag}, which remains False.
        
        \item The function returns "Sad" because three consecutive vowels were not found. The result "Sad" is printed to the console, which is also our expected output

    \end{itemize}
    
\end{enumerate}

\alert{Green} {This means the applied logic is correct}






    \titledquestion{Game enjoyer}
    You are playing a video game, and are now fighting the final boss.

    The boss has $H$ HP (health points), each normal attack of your character reduces the boss health by $X$.

    Your character also has a special attack that can only be used once during the boss fight, and it will decrease the health of the boss by Y.

    You win the battle when the health of the boss is less than 0.

    Write a function minattacks that returns the minimum number of attacks needed by your character to win the boss battle.

    Make sure to call your function and to present the output use:

    print(functionname(parameters))

    \heading{Constraints}
    \begin{itemize}
        \item $ 1 \leq X < Y \leq H \leq 100 $

    \end{itemize}

    \heading{Interaction}

    The input comprises a single line containing 3 space-separated integers denoting the values of $H, X, Y$

    The output must contain a number denoting the minimum attacks.

    \heading{Sample}

    \rowcolors{2}{gray!25}{white}
    \begin{tabularx}{\textwidth}{|X|X|}
        \rowcolor{gray!50}
        \hline
        Input     & Output \\ \hline\hline
        100 25 40 & 4      \\\hline
        100 29 45 & 3      \\\hline
    \end{tabularx}

    In the first case, $(H,X,Y)=(100,25,40)$. Your character can attack the boss 4 times normally and that is enough kill the boss. (you can also use special attack and the total attacks would still be 4)

    In the second case, $(H,X,Y)=(100,29,45)$. Your character can attack the boss 2 times normally and then use the special attack dealing 103 damage in total.

    \heading{Proposed Solution}
    \begin{Verbatim}[gobble =1 ,numbers=left,xleftmargin=5mm, numbersep=10pt, commandchars=\\\{\}]
 def minattacks(H,X,Y):
    A = H - Y
    if A % X == 0:
        return A // X + 1 
    return A / X + 2
    print(minattacks(X,Y,H))
    \end{Verbatim}

    \heading{Dry Run}\\
    Using one of the inputs provided in the Sample section above, dry run the proposed code solution below.


With $H=100$, $X=25$, and $Y=40$, the code runs as follows:

\begin{enumerate}
\item Inside the function, it calculates $A$ as $H - Y$, which is $100 - 40$, resulting in $A = 60$.

\item It checks if $A$ is divisible by $X$ with the condition $A \% X == 0$. In this case, $60 \% 25$ is not equal to $0$, so this condition is False.

\item Since the condition is False, the program resumes from line 5 and returns $A / X + 2$. Since $A = 60$ and $X = 25$, it calculates $60 / 25 + 2$, which equals $2.4 + 2 = 4.4$.

\item The function will return $4.4$, which is not our expected output.

\item The function will also not be called as the function call is unreachable (indented within the function definition)
\end{enumerate}

    
\heading{Error Identification}\\
    Briefly explain the errors you identified in the proposed code solution. Mention the line number and the errors in each line.

\begin{itemize}
    \item Line 5: Floating-point result through simple division (/) rather than floor division (//).
    \item Line 6: Unreachable function call due to indentation error.
\end{itemize}

  \heading{Correct Solution}\\
  Rewrite the lines of code you mentioned above with their errors corrected.
  \begin{verbatim}
 def minattacks(H,X,Y):
    A = H - Y
    if A % X == 0:
        return A // X + 1 
    return A // X + 2
 print(minattacks(X,Y,H))
  \end{verbatim}
  % New sheet for rough work
  \newpage
  \centerline{\heading{Rough Work}}

\end{questions}

\end{document}

%%% Local Variables:
%%% mode: latex
%%% TeX-master: t
%%% End:
