%%%%%%%%%%%%%%%%%%%%%%%%%%%%% Define Article %%%%%%%%%%%%%%%%%%%%%%%%%%%%%%%%%%
\documentclass{article}
%%%%%%%%%%%%%%%%%%%%%%%%%%%%%%%%%%%%%%%%%%%%%%%%%%%%%%%%%%%%%%%%%%%%%%%%%%%%%%%

%%%%%%%%%%%%%%%%%%%%%%%%%%%%% Using Packages %%%%%%%%%%%%%%%%%%%%%%%%%%%%%%%%%%
\usepackage{geometry}
\usepackage{graphicx}
\usepackage{amssymb}
\usepackage{amsmath}
\usepackage{amsthm}
\usepackage{empheq}
\usepackage{mdframed}
\usepackage{booktabs}
\usepackage{lipsum}
\usepackage{graphicx}
\usepackage{color}
\usepackage{psfrag}
\usepackage{pgfplots}
\usepackage{bm}
\usepackage{wrapfig}
\usepackage{hyperref}
\usepackage{url}
%%%%%%%%%%%%%%%%%%%%%%%%%%%%%%%%%%%%%%%%%%%%%%%%%%%%%%%%%%%%%%%%%%%%%%%%%%%%%%%

% Other Settings

%%%%%%%%%%%%%%%%%%%%%%%%%% Page Setting %%%%%%%%%%%%%%%%%%%%%%%%%%%%%%%%%%%%%%%
\geometry{a4paper}

%%%%%%%%%%%%%%%%%%%%%%%%%% Define some useful colors %%%%%%%%%%%%%%%%%%%%%%%%%%
\definecolor{ocre}{RGB}{243,102,25}
\definecolor{mygray}{RGB}{243,243,244}
\definecolor{deepGreen}{RGB}{26,111,0}
\definecolor{shallowGreen}{RGB}{235,255,255}
\definecolor{deepBlue}{RGB}{61,124,222}
\definecolor{shallowBlue}{RGB}{235,249,255}
%%%%%%%%%%%%%%%%%%%%%%%%%%%%%%%%%%%%%%%%%%%%%%%%%%%%%%%%%%%%%%%%%%%%%%%%%%%%%%%

%%%%%%%%%%%%%%%%%%%%%%%%%% Define an orangebox command %%%%%%%%%%%%%%%%%%%%%%%%
\newcommand\orangebox[1]{\fcolorbox{ocre}{mygray}{\hspace{1em}#1\hspace{1em}}}
%%%%%%%%%%%%%%%%%%%%%%%%%%%%%%%%%%%%%%%%%%%%%%%%%%%%%%%%%%%%%%%%%%%%%%%%%%%%%%%

%%%%%%%%%%%%%%%%%%%%%%%%%%%% English Environments %%%%%%%%%%%%%%%%%%%%%%%%%%%%%
\newtheoremstyle{mytheoremstyle}{3pt}{3pt}{\normalfont}{0cm}{\rmfamily\bfseries}{}{1em}{{\color{black}\thmname{#1}~\thmnumber{#2}}\thmnote{\,--\,#3}}
\newtheoremstyle{myproblemstyle}{3pt}{3pt}{\normalfont}{0cm}{\rmfamily\bfseries}{}{1em}{{\color{black}\thmname{#1}~\thmnumber{#2}}\thmnote{\,--\,#3}}
\theoremstyle{mytheoremstyle}
\newmdtheoremenv[linewidth=1pt,backgroundcolor=shallowGreen,linecolor=deepGreen,leftmargin=0pt,innerleftmargin=20pt,innerrightmargin=20pt,]{theorem}{Theorem}[section]
\theoremstyle{mytheoremstyle}
\newmdtheoremenv[linewidth=1pt,backgroundcolor=shallowBlue,linecolor=deepBlue,leftmargin=0pt,innerleftmargin=20pt,innerrightmargin=20pt,]{definition}{Definition}[section]
\theoremstyle{myproblemstyle}
\newmdtheoremenv[linecolor=black,leftmargin=0pt,innerleftmargin=10pt,innerrightmargin=10pt,]{problem}{Problem}[section]
%%%%%%%%%%%%%%%%%%%%%%%%%%%%%%%%%%%%%%%%%%%%%%%%%%%%%%%%%%%%%%%%%%%%%%%%%%%%%%%

%%%%%%%%%%%%%%%%%%%%%%%%%%%%%%% Plotting Settings %%%%%%%%%%%%%%%%%%%%%%%%%%%%%
\usepgfplotslibrary{colorbrewer}
\pgfplotsset{width=8cm,compat=1.9}
%%%%%%%%%%%%%%%%%%%%%%%%%%%%%%%%%%%%%%%%%%%%%%%%%%%%%%%%%%%%%%%%%%%%%%%%%%%%%%%

%%%%%%%%%%%%%%%%%%%%%%%%%%%%%%% Title & Author %%%%%%%%%%%%%%%%%%%%%%%%%%%%%%%%
\title{AI Career Exploration and Skill Gap Analysis}
\author{Ali Muhammad Asad \\ aa07190}
\date{} %Leave uncommented if u want automatic date which is done through maketitle, else u can uncomment this and type anything else u want over here - not necessary to enter a date over here
%%%%%%%%%%%%%%%%%%%%%%%%%%%%%%%%%%%%%%%%%%%%%%%%%%%%%%%%%%%%%%%%%%%%%%%%%%%%%%%

\begin{document}
    \maketitle

\subsection*{1. Job Description}

\textbf{Job Title:} AI Safety Engineer - Link can be found \href{https://boards.greenhouse.io/xai/jobs/4531703007}{here.}

\noindent \textbf{Location:} Bay Area - San Francisco and Palo Alto

\noindent \textbf{Expires:} 04/30/2026

\noindent \textbf{xAI} is dedicated to creating AI systems that can accurately understand the universe and aid humanity in its pursuit of knowledge. The AI safety engineer role focuses on identifying safety shortcomings in their AI system, Grok, and implementing methods from machine learning safety literature to enhance its safety. This involves designing evaluations to assess potential risks, collaborating with other teams to integrate safety measures, and innovating new ideas to develop AI systems that can accurately understand the universe.

\subsection*{2. Skills Breakdown}

\textbf{Tech Stack:}
\begin{itemize}
    \item Python
    \item JAX and XLA
    \item Rust
    \item Cuda (C++ and Triton)
\end{itemize}

\noindent \textbf{Key Responsibilities:}
\begin{itemize}
    \item Identify shortcomings in Grok
    \item Adapt existing or design new evaluations to assess potential risks
    \item Implement methods from ML safety literature to improve Grok's safety
    \item Collaborate with other teams to ensure safety measures are integrated into the development pipeline
    \item Innovate new ideas to advance the goal of developing AI systems that understand the universe
\end{itemize}

\noindent \textbf{Required Skills and Qualifications:}
\begin{itemize}
    \item Experience working on ML safety problems with demonstrated exceptional work
    \item Expertise in machine learning and fine-tuning large language models
    \item Experience in developing and managing large-scale machine learning systems
    \item Ability to keep up with multiple ML safety domains
\end{itemize}

\noindent \textbf{Soft Skills:} Strong communications skills, hands-on approach to contribute directly to the company's mission. 

\subsection*{3. Self Assessment}

\noindent Throughout my degree, I've taken up various courses in AI and machine learning, including Computational Intelligence, Deep Learning, Probabilistic Graphical Models, Reinforcement Learning, Large Language Models, and currently Generative AI: Security, Ethics and Governance. Thus, I have engaged in AI and ML projects, including my Final Year Project which basically is developing a Large Language Model from scratch for a low resource language which would include evaluation, assessment, and safety measures. My projects have also been aligned with the company's mission of developing AI systems that can accurately understand the universe and aid humanity, such as in the preservation of literature (deep learning), predicting guide efficacy of Crispr-Cas systems (probabilistic graphical models) that can aid biologists in their research for effective gene editing and drug discovery, and developing a large language model for a low resource language, which can aid the people of that region in their communication and understanding of the world. Therefore, I am well versed in Python, however, I do not have experience with JAX and XLA, Rust, or Cuda. I am willing to learn these languages and tools to adapt to the company's tech stack. I have experience in developing and managing large-scale machine learning systems, and I am confident in my ability to adapt to multiple ML safety domains. I have strong communication skills and a hands-on approach to contribute directly to the company's mission.

\subsection*{4. Knowledge \& Skills Gap}

I do not have prior experience with JAX and XLA, Rust, or Cuda. I will need to learn these languages and tools to adapt to the company's tech stack. I will also need to gain more experience in ML safety problems and fine-tuning large language models to meet the company's requirements - which I will have hopefully done by the end of my Generative AI course, and towards the conclusion of my Final Year Project. I am also not familiar with Triton, and will need to learn this tool to adapt to the company's tech stack. Grok is also something I am not familiar with, and I will need to learn about this system to identify its shortcomings and implement methods from ML safety literature to improve its safety. 

\subsection*{5. Action Plan for Filling the Gaps}

To learn the above-mentioned languages and tools, I will take online courses and tutorials, and work on projects that require the use of these languages and tools. I will also read research papers and articles on ML safety problems and fine-tuning large language models to gain more experience in these areas. I will also read up on Triton and Grok to familiarize myself with these systems.


\subsection*{6. Reflection}

The exercise to find an AI job description and analyze my skills and knowledge gap has been very insightful. It has helped me identify areas where I need to improve and develop my skills further. I am excited to take on the challenge of learning new languages and tools, and gaining more experience in ML safety problems, something which I have not done, neither considered before. I am confident that I will be able to fill the gaps in my knowledge and skills and be able to apply for the job in the future.

\end{document}